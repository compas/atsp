\documentclass[fleqn,10pt]{book}
\usepackage{a4wide,alltt,moreverb}
\usepackage{t1enc}
\usepackage[final]{graphicx}
\usepackage{psfrag}
%\usepackage[swedish]{babel}
%\selectlanguage{swedish}
\usepackage{amsmath,amssymb,latexsym,theorem}
\input{epsf}
\thispagestyle{empty}
\newcommand{\HRule}{\rule{\linewidth}{1mm}}
\setlength{\parindent}{0mm}
\setlength{\parskip}{0mm}


\begin{document}
\vspace*{\stretch{1}}
\HRule
\begin{flushright}
\huge A practical guide to {\sc Atsp}2K\\[7mm]
\huge P. J\"onsson\\
\end{flushright}
\HRule

\vspace{8 mm}

\vspace*{\stretch{2}}
\begin{center}
\Large \textsc{COMPutational Atomic Structure Group 2016}
\end{center}

\clearpage



\tableofcontents 
\clearpage

\noindent
\chapter{The ATSP2K package}
\section{To compile}
It you install this on some cluster you may have to load the \verb+openmpi+ module. On the Malm\"o cluster this is done by
\verb+ module add openmpi/3.0.0_gfortran+. To see which modules are availabler do \verb+module avail+.\medskip\\
Edit the \verb+Install_gfortran+ file and set the \verb+MPI_TMP+ directory where all the MPI files should reside.\medskip\\
Delete all the files in the bin and lib directories.\medskip\\
Issue the commands
\begin{verbatim}
     source ./Install_gfortran
     cd src
     make clean
     make
\end{verbatim}
Upon successful compilation the executable programs and tools can be found in the bin directory. Programs running in parallel under MPI have extensions mpi. Make sure to put the executables on the path.
\section{To run MPI codes}
Upon the start of each new session with the MPI codes \verb+MPI_TMP+ environment variable MUST be set.
This can be done by issuing the command 
\begin{verbatim}
     source ./Install_gfortran
\end{verbatim}
Before each new run of the \verb+nonh_mpi+ and \verb+bpci_mpi+ programs (not the \verb+mchf_mpi+ program) the \verb+MPI_TMP+ directory with all the temporary MPI files must be cleaned.\medskip\\
The MPI codes are run by issuing the command \verb+mpirun -np+ and giving the number of nodes. For example,
to run the \verb+nonh_mpi+ code on 8 nodes give the command
\begin{verbatim}
     mpirun -np 8 nonh_mpi
\end{verbatim}
\section{Application programs and tools}
The new version of the {\sc Atsp}2K package [1] consists of a number of application programs and tools. The application programs and tools, along with the underlying theory, are described in the original
write-ups [2-13]. See also the book Computational Atomic Structure by Froese Fischer, Brage, and J\"onsson [14].\medskip\\
Below is a partial list of programs in the package: 


\begin{enumerate}
\item Routines that generate a configuration state list (CSL):
\begin{enumerate}
\item {\tt gencl}       -- generate a configuration state list (CSL) 
from lists of reference configurations 
\item {\tt lsgen}     -- generate a CSL using rules
\item {\tt csfexcitation}     -- generate a CSL using rules (preferred program)
\item {\tt lsreduce}  -- include only CSFs that have at least one non-zero 
matrix element with a CSF of a reference list
\end{enumerate}

\item {\tt nonh}       -- compute angular coefficients
\item {\tt nonh\_mpi}       -- compute angular coefficients, MPI version 
\item {\tt nonhz}  -- compute angular coefficients assuming a partition of the configuration state list into zero- and first order spaces (optional)    

\item {\tt mchf}      -- determine orbitals and mixing coefficients
\item {\tt mchf\_mpi}      -- determine orbitals and mixing coefficients, MPI version
\item {\tt mchf\_C}      -- determine orbitals and mixing coefficients, assuming {\tt c.lst} on disc (medium size calculations, to be used ONLY if ordinary {\tt mchf} fails on current operating system) 

\item {\tt mchf\_HC}      -- determine orbitals and mixing coefficients, assuming {\tt c.lst} and {\tt hmx.lst} (large size calculations, to be used ONLY if ordinary {\tt mchf} fails on current operating system) 

\item {\tt bpic}      -- perform Breit-Pauli CI calculation, where matrix elements computed on the fly
\item {\tt bpic\_mpi}      -- perform Breit-Pauli CI calculation, where matrix elements computed on the fly, MPI version.
\item {\tt bp\_ang + bp\_mat + bp\_eiv}      -- perform Breit-Pauli CI calculation, by separating angular integration and saveon disc, the construct matrix and diagonalize (fine eigenvalues). Suitable in smaller cases for an iso-electronic sequence.

\item Routines for computing transition probabilities:
\begin{enumerate}
\item  {\tt trans}     -- perform transition calculations in a single
 orthonormal basis 
\item {\tt biotr} -- perform biorthogonal transformation and subsequent transition calculation 
between states.
\end{enumerate}

\item {\tt hfs}      -- compute hyperfine interactions 

  \end{enumerate}

A number of generally 
short programs have been developed as tools to facilitate a computational
procedure. 

\begin{enumerate}

\item {\tt levels}: 
\item {\tt comp}:  Given the {\tt name} of a set of files, and a tolerance, this program displays the 
composition of each state in the file by listing the  CSFs and their expansion coefficients in order of
 decreasing magnitude provided the magnitude is greater than the tolerance.
 The {\tt name.c} option usually may not apply (seen {\tt condens} below).
 This composition information may be of importance in determining the multi-reference 
set or the extent of correlation.  It is also essential for resolving naming conflicts.

\item {\tt condens}: Given the {\tt name} of a case,
the source for the expansion coefficients (usually .l or .j) and whether results are to be sorted,
 the program produces a file {\tt cfg.out} that includes only those CSFs for which  $\sqrt{\sum_i c_i^2}
 \ge T$, where $T$ is the cut-off tolerance. 
 Since {\tt mchf} no longer produces a {\tt cfg.out} with the
expansion coefficients included, the {\tt name.c} option cannot usually be applied.
  The coefficients are included in 
the condensed output file, {\tt cfg.out}. 
  To restore this file to the present .c format (without expansion coefficients), 
  move the file to {\tt clist.inp} and run {\tt lsgen} 
selecting the ``r'' mode (for restore).  Note that condensing may destroy the ``closed under 
de-excitation'' requirement of  {\tt biotr} but when the tolerance is sufficiently small, no problems have been encountered. 
In a systematic method, where
maximum principal quantum number is increased by unity from one iteration to 
the next, it is unlikely that a CSF with a high $n$ will remain and the
equivalent CSF with a lower $n$ be deleted.






\item {\tt tables}:  This is a program written in C++ that takes a {\tt name.lsj} file, usually a concatenated file of all the {\tt .lsj} transition files 
for a given atom or ion,
and finds the energy level structure of the levels and the multiplet
transition arrays. The tables posted at 
the website {\tt http://atoms.vuse.vanderbilt.edu} are examples of tables
produced by the {\tt tables} program.
When an energy level is present in the {\tt .lsj} file with {\bf two} different 
energies, the higher level is considered to be unphysical.  It and all data 
associated with this level are removed from the table.  The user may also 
specify certain levels as ``unphysical'' in which case they will be removed. 
 Finally, the program computes the lifetimes of
 the levels from the transition data provided in the file.

\item {\tt T\_dependence}:  Given a {\tt .c} and a corresponding {\tt .j}
 file, this program displays the term dependence of each state included in the
 file. This gives an 
indication of $LS$ term mixing in the wave function of a state.  
The Breit-Pauli 
programs all assign labels to states according to the largest expansion 
coefficient. When this process produces the same label for two different states 
a careful analysis is needed. The $LS$ value should be that term 
with the largest 
composition and within that $LS$, the largest expansion coefficient identifies 
the label.  For more on labeling, see for example
Fischer, C.F.; Tachiev, G. Breit-Pauli energy levels, lifetimes, and transition probabilities for the beryllium-like to neon-like sequences. At. Data Nucl. Data Tables 2004, 87, 1-184 and 
Fischer, C.F.; Gaigalas, 
G. Multiconfiguration Dirac-Hartree-Fock energy levels and transition probabilities for W XXXVIII. Phys. Rev. A 2012, 85, 042501.
\item {\tt relabel} This utility reads radial functions in turn from {\tt wfn.inp}. 
 For each function, the user may enter either a blank, ``d'', or a 
new 3-character label for which the respective action is to write the radial 
function unchanged
into {\tt wfn.out}, skip (or delete) the radial function, write the radial function 
changing the displayed label to the entered label.

\item {\tt select}  This routine selects those CSFs from a designated list that
contain orbitals specified by the user.  This may be useful when expansions that
have been condensed are extended with new CSFs to be included to first-order.
The selected CSFs may be appended to the condensed list. 

\item {\tt w\_format and w\_unformat}: Binary file formats are not always compatible 
in going from one system environment to another. {\tt w\_format} 
takes a {\tt wfn.inp} file (in .w format) and produces a formatted
{\tt wfn.fmt} file.  The program {\tt w\_unformat} reads {\tt wfn.fmt}
and produces the binary file (in .w format) {\tt wfn.out}.  Some loss of accuracy can be 
expected in the process but the radial functions are tabulated to 11-digits of accuracy.
\end{enumerate} 
\section*{References}
\begin{enumerate} 
\item ATSP2K:  C. Froese Fisc
her, G. Tachiev, G. Gaigalas, M.R. Godefroid
Comput. Phys. Commun.
  176 559 (2007)
\item MCHF\_LIBRARIES: C.Froese Fischer, 	Comput. Phys. Commun.
 64 399 (1991)
\item MCHF\_ GENCL: C. Froese Fischer, B. Liu, Comput. Phys. Commun.
 64 406 (1991)
\item MCHF\_ NONH: A. Hibbert, C. Froese Fischer, 	    Comput. Phys. Commun.
 64 417 (1991)
\item MCHF\_88: C. Froese Fischer,  	Comput. Phys. Commun.
 64 431 (1991)
\item MCHF\_BREIT: A. Hibbert, R. Glass, C. Froese Fischer 	Comput. Phys. Commun.
 64 455 (1991)
\item MCHF\_CI: C. Froese Fischer, 	Comput. Phys. Commun.
 64 473 (1991)
\item MCHF\_MLTPOL: C. Froese Fischer, M.R. Godefroid, A. Hibbert, 	Comput. Phys. Commun.
 64 486 (1991)
\item MCHF\_LSTR AND MCHF\_LSJTR: C.Froese Fischer, M.R. Godefroid, 	Comput. Phys. Commun.
 64 501 (1991)
\item MCHF\_AUTO: C.Froese Fischer, T. Brage, 	Comput. Phys. Commun.
 74 381  (1993)
\item MCHF\_HFS: P. J\"onsson, C.-G. Wahlstrom, C.Froese Fischer, 	Comput. Phys. Commun.
 74 399 (1993)
\item MCHF\_ISOTOPE: C. Froese Fischer, L. Smentek-Mielczarek, N. Vaeck, G. Miecznik, 	Comput. Phys. Commun.
 74 415 (1993)
\item MCHF\_LSGEN: L. Sturesson and C. Froese Fischer, Comput. Phys. Commun. 74 432 (1993)
\item C. Froese Fischer, T. Brage, P. J\"onsson, Computational Atomic Structure - an MCHF approach, IoP, (1997)
\end{enumerate}

\clearpage

\section{File naming conventions, program and data flow}
Passing of information between different programs is done through files. This process is greatly facilitated through file
naming conventions.  A name is associated with the results and an extension that defines the contents and format
of the file. Thus the file name becomes \verb+name.extension+. Common extensions are listed in Table 1.1.\medskip\\
To run
{\sc Atsp}2K a number of programs need to be run in a pre-determined sequence. Figure 1.1
displays a typical sequence of block version program calls to evaluate different expectation values.
The resulting flow of files is displayed in Figure 1.2. 

\begin{table}[h!]
\caption{
\label{tab:extension}
 Table of common extensions.}
\begin{center}
\begin{tabular}{l l}
{\sl extension} & {\sl data in the file}\\
\hline
{\tt .c} &  configuration state function (CSF) expansion \\
{\tt .w} &  radial wave functions (numerical values in binary form)\\
{\tt .l} &  expansion coefficients from a non-relativistic ($LS$) calculation\\
{\tt .j} &  expansion coefficients from a Breit-Pauli ($LSJ$) calculation \\
{\tt .h} &  hyperfine data \\
{\tt .t} &  term dependence of a .j file\\
{\tt .ls} &  transition probability data a non-relativistic ($LS$) calculation \\
{\tt .lsj} & transition probability data a Breit-Pauli ($LSJ$) calculation \\
\hline
\end{tabular}
\end{center}
\end{table}

\begin{figure}
\fbox{
\begin{picture}(450,380)(0,150)
\thicklines
\put (100,500){{\tt hf}}
\put (150,500){(Generation of initial radial wave functions)}
\put (110,480){\vector(0,-1){10}}
\put (100,450){{\tt lsgen (or csfexcitation)}}
\put (230,450){(Generation of configuration state list)}

\put (110,430){\vector(0,-1){10}}
\put (100,400){{\tt lsreduce}}
\put (150,400){(Reduction of configuration state list)}

\put (110,380){\vector(0,-1){10}}
\put (100,350){{\tt nonh}}
\put (150,350){(Angular integration)}
\put (110,330){\vector(0,-1){10}}
\put (100,300){{\tt mchf}}
\put (150,300){(Self-consistent field procedure)}
\put (110,280){\vector(0,-1){10}}

\put (100,250){{\tt bpci}}
\put (150,250){(Breit-Pauli CI calculation)}
\put (110,230){\vector(0,-1){10}}

\put (100,200){{\tt biotr}}
\put (150,200){(Transition rate calculation)}
%\put (110,180){\vector(0,-1){10}}

\put (100,175){{\tt hfs}}
\put (150,175){(Hyperfine structure calculation)}





\end{picture}
}
\caption{
 \label{fig:flow_blk}
 Typical sequence of block version program calls 
 to evaluate different expectation values such as transition rates and hyperfine structure constants.}
\end{figure}


\begin{figure}
{\small 
\fbox{
\begin{picture}(450,370)(0,195)
\thicklines
\put(30,535){{\tt hf}} \put (40,530){\vector(0,-1){10}}
\put (85,535){Output; {\tt wfn.out}}

\put (30,495){{\tt lsgen (or csfexcitation)}} \put (40,490){\vector(0,-1){10}}
\put (155,495){Output; {\tt clist.out}}


\put(30,455){{\tt lsreduce}} \put (40,435){\vector(0,-1){10}}
\put(85,455){Input; {\tt mrlist, cfg.inp}}
\put(85,440){Output; {\tt cfg.out}}

\put(30,415){{\tt nonh}} \put (40,395){\vector(0,-1){10}}
\put(85,415){Input; {\tt cfg.inp}}
\put(85,400){Output; {\tt number of different angular files}}

    \put(30,365){{\tt mchf}} \put(40,345){\vector(0,-1){10}}
\put(85,365){Input; {\tt cfg.inp, wfn.inp}}
\put(85,350){Output; {\tt LS}$\pi${\tt.l}, {\tt wfn.out, summry}}

    \put(30,315){{\tt bpci}} \put(40,295){\vector(0,-1){10}}
\put(85,315){Input; {\tt name.c, name.w}}
\put(85,300){Output; {\tt name.j}}

    \put(30,265){{\tt biotr}} 
\put(85,265){Input; {\tt name1.c, name1.w, name1.j, name2.c, name2.w, name2.j}}
\put(85,250){Output; {\tt name1.name2.lsj}}

\put(30,225){{\tt hfs}} 
\put(85,225){Input; {\tt name.c}, {\tt name.w}, {\tt name.l}}
\put(85,210){Output; {\tt name.h}} 
\end{picture}
}}
\caption{
 \label{fig:dataflow_blk}
 Flow of files for a normal sequence of program runs.}
\end{figure}

\clearpage

\section{Generating lists of configuration state functions}
Exploring different correlation models and generating lists of configuration state functions (CSFs) is a major task of the computation. The {\sc Atsp}2K provides 
several programs for performing this task. For generating small lists of CSFs it is often best to use the \verb+gencl+ program. To generate expansion based on the notion of excitations from subshells to an active set of orbitals it is often advantageous to use the \verb+lsgen+ program. Different restrictions can be put on the excitations and it is possible to describe core-valence correlation where at most one excitation is allowed from subshells of the core. To make sure that the generated CSFs interacts with the CSFs in multireference the program \verb+lsreduce+ should be used.
Before continuing the reader is advised to study the write-up of the \verb+lsgen+ program [L. Sturesson and C. Froese Fischer, Comput. Phys. 
Commun. 74, 432 (1993)]. To simplify the use of \verb+lsgen+ program the program \verb+csfexcitation+ has been created (see section 3.1 for the use of this program).
\section{Spectroscopic orbitals}
The ``spectroscopic orbitals'' are those where node counting is
required to ensure that the self-consistent field procedure converges to the
desired solution [C. Froese Fischer, T. Brage, P. J\"onsson, Computational Atomic Structure - an MCHF approach, IoP, 1997]. Spectroscopic orbitals 
build the reference CSFs and often have occupation numbers near unity or more.  All other orbitals are ``correlation
orbitals''. If the self-consistent field procedure fails for spectroscopic orbitals, i.e. the wrong number of nodes are obtained with a subsequent program halt. 
\chapter{Running the application programs} 
In this chapter we demonstrate the use of the application programs of {\sc Atsp}2K in three cases described below. The use of the tools of the {\sc Atsp}2K package is described in the next chapter. The data written to the output files are explained and discussed in chapter 4. 
\section{First example: $1s^22s~^2S$ and $1s^22p~^2P$ in Li I}
The first example is for $1s^22s~^2S_{1/2}$ and $1s^22p~^2P_{1/2,3/2}$ in Li. 
\subsection*{Overview}
\begin{enumerate}
\item Perform HF calculation for $1s^22s~^2S$.
\item Generate $n = 3$ CAS configuration list for $1s^22s~^2S$.
\item Perform angular integration.
\item Perform self-consistent field calculation.
\item Save output to \verb+2S_CAS_3+. 
\item Calculate hfs.
\item Generate $n = 3$ CAS Breit configuration list for $1s^22s~^2S_{1/2}$.  
\item Perform CI Breit-Pauli calculation 
\item Perform HF calculation for $1s^22p~^2P$.
\item Generate $n = 3$ CAS configuration list for $1s^22p~^2P$.
\item Perform angular integration.
\item Perform self-consistent field calculation.
\item Save output to \verb+2P_CAS_3+. 
\item Calculate hfs.
\item Generate $n = 3$ CAS Breit configuration list for $1s^22p~^2P_{1/2,3/2}$.  
\item Perform CI Breit-Pauli calculation 
\item Compute the transition rates from the CI wave functions.
\end{enumerate} 

\subsection*{Program input}
In the test-runs input is marked by \verb+>>+ and \verb+>>3+, for example, indicate that the user should input 3 and then strike the return key.
When \verb+>>+ is followed by blanks just strike the return key.
\begin{verbatim}

*******************************************************************************
*          RUN HF FOR 1s(2)2s 2S                                              *
*          OUTPUT FILE: wfn.out, hf.log                                       *
*******************************************************************************

>>HF

                      =============================
                       H A R T R E E - F O C K . 96
                      =============================


               THE DIMENSIONS FOR THE CURRENT VERSION ARE:
                          NWF= 20        NO=220



 START OF CASE
 =============


 Enter ATOM,TERM,Z
 Examples: O,3P,8. or Oxygen,AV,8.
>>Li,2S,3.

 List the CLOSED shells in the fields indicated (blank line if none)
 ... ... ... ... ... ... ... ... etc.
>>1s

 Enter electrons outside CLOSED shells (blank line if none)
 Example: 2s(1)2p(3)
>>2s(1)

 There are   2 orbitals as follows:
   1s  2s

 Orbitals to be varied: ALL/NONE/=i (last i)/comma delimited list/H
>>all

 Default electron parameters ? (Y/N/H)
>>y

 Default values for remaining parameters? (Y/N/H)
>>y
 
 
 ..................
 
          ITERATION NUMBER  7
          ----------------

          SCF CONVERGENCE CRITERIA (SCFTOL*SQRT(Z*NWF)) =   1.6D-06

          C( 1s 2s) =     0.00000   V( 1s 2s) =    -2.19991   EPS = 0.000000
          E( 2s 1s) =     0.00573   E( 1s 2s) =     0.00286

                     EL         ED             AZ           NORM       DPM
                     1s      4.9554830      9.2603703   1.0000000    1.43D-08
                     2s      0.3926457      1.4468006   0.9999999    1.28D-08


      < 1s| 2s>= 8.7D-09


     TOTAL ENERGY (a.u.)
     ----- ------
           Non-Relativistic       -7.43272693    Kinetic        7.43272693
           Relativistic Shift     -0.00054376    Potential    -14.86545385
           Relativistic           -7.43327069    Ratio        -2.000000000

 Additional parameters ? (Y/N/H)
>>n

 Do you wish to continue along the sequence ?
>>n


 END OF CASE
 ===========
 
*******************************************************************************
*         COPY FILES                                                          *
*         Keep a copy of the wfn.out in 2SeDF.w file for reference            *
*         Keep a copy of hf.log in 2SDF.log                                   *
*******************************************************************************

cp wfn.out wfn.inp
cp wfn.out 2SeDF.w
cp hf.log 2SeDF.log

*******************************************************************************
*         RUN LSGEN TO GENERATE N = 3 CAS CONFIGURATION LIST FOR 2S           *
*         OUTPUT FILES: clist.out, clist.log                                  *
*         TO USE CSFEXCITATION SEE SECTION 3.1!!!!!                           *
*******************************************************************************

 New list, add to existing list, expand existing list, optimized sorting, restored order or quit? (*/a/e/s/r/q)
>>
 Breit or MCHF? (B/*)
>>
 Default, symmetry or user specified ordering? (*/s/u)
>>
 Highest principal quantum number, n? (1..15)
>>3
 Highest orbital angular momentum, l? (s..d)
>>d
 Are all these nl-subshells active? (n/*)
>>
 Limitations on population of n-subshells? (y/*)
>>
 Highest n-number in reference configuration? (1..3)
>>2
 Number of electrons in 1s? (0..2)
>>2
 Number of electrons in 2s? (0..2)
>>1
 Number of electrons in 2p? (0..6)
>>0
 Resulting term? (1S, 3P, etc.)
>>2S
 Number of excitations = ? (0..3)
>>3
 27 configuration states have been generated.
 Generate a second list? (y/*)
>>
 27 configuration states in the final list.
 The generated file is called clist.out.
FORTRAN STOP

*******************************************************************************
*         COPY FILES                                                          *
*         IT IS ADVISABLE TO SAVE THE LSGEN LOG-FILE TO HAVE A RECORD ON      *
*         HOW THE CONFIGURATION LISTS GENERATION WAS DONE                     *
*******************************************************************************

>>cp clist.log 2SeCAS3.log
>>cp clist.out cfg.inp

*******************************************************************************
*         RUN NONH TO GENERATE ENERGY EXPRESSION                              *
*         INPUT FILES: cfg.inp                                                *
*         OUTPUT FILES: cfg.h, yint.lst, c.lst, ih.=n.lst                     *
*******************************************************************************

>>nonh

 input file is cfg.inp ...



                       ===============================
                             N O N H       2000
                       ===============================



 THERE ARE  6 ORBITALS AS FOLLOWS:

       1s  2s  2p  3s  3p  3d

 THERE ARE  0 CLOSED SUBSHELLS COMMON TO ALL CONFIGURATIONS AS FOLLOWS:


 Allocating space for           214  integrals
 processing 2Se with            27 configurations
     277 non-zero matrix elements
      92 NF      76 NG     446 NR     109 NL
     723 Total number of integrals

 end-of-file clist!!!

*******************************************************************************
*         RUN MCHF TO OBTAIN SELF CONSISTENT SOLUTIONS                        *
*         INPUT FILES: wfn.inp (optional), cfg.inp, angular files             *
*         OUTPUT FILES: wfn.out, 2Se.l, summry                                *
*                                                                             *
*         NOTE1: Mixing coefficients written to a file named by LS-symmetry   *
*         and parity, in this case 2Se.l                                      *
*                                                                             *
*         NOTE2: We force the program to iterate to higher precision than     *
*         default which is a good thing for small calculations                *
*******************************************************************************

>>mchf

                    =======================
                            M C H F  ... 2000
                    =======================




          THE DIMENSIONS FOR THE CURRENT VERSION ARE:
             NWD= 60        NO=220    Lagrange Multipliers=800



 START OF CASE
 =============


 ATOM, Z in FORMAT(A, F) :
>>Li,3.
 cfg.inp has configurations for             1  terms

 Enter eigenvalues and weights: one line per term, eigenvalues with weights
 in parenthesis and separated by commas, default is 1.0
 2Se
>>1

 There are   6 orbitals as follows:
   1s  2s  2p  3s  3p  3d
 Enter orbitals to be varied: (ALL,NONE,SOME,NIT=,comma delimited list)
>>all
 Enter those that are spectroscopic
>>1s,2s

 Default electron parameters ? (Y/N)
>>y

 Default values for other parameters? (Y/N)
>>n

 Default values (NO,REL,STRONG) ? (Y/N)
>>y

 Li        3.   220  6  6  0  F

 Default values for PRINT, CFGTOL, SCFTOL ? (Y/N)
>>n
 Input free FORMAT(L, F, F)
>>.f.,1.e-50,1.e-50

 Default values for NSCF, IC ? (Y/N)
>>n
 Input free FORMAT(I, I)
>>300,0

 Default values for ACFG,LD,TRACE? (Y/N)
>>y

 .................
 
          ITERATION NUMBER 111
          ----------------

          CONVERGENCE CRITERIA:ENERGY  (CFGTOL)            =  1.0D-50
                              :FUNCTION(SCFTOL*SQRT(Z*NWF))=  4.2D-50

          E( 2s 1s) =     0.02212   E( 1s 2s) =     0.01109
          E( 3s 1s) =    28.98616   E( 1s 3s) =     0.03558
          E( 3s 2s) =     4.20140   E( 2s 3s) =     0.01028
          E( 3p 2p) =    51.03003   E( 2p 3p) =     5.25729

                     EL         ED             AZ           NORM       DPM
      < 1s| 2s>= 4.2D-11
      < 1s| 3s>=-4.5D-10
                     1s      5.0342669      9.2695740   1.0000000    5.17D-10
      < 2s| 3s>=-7.3D-09
      < 1s| 2s>= 1.0D-08
                     2s      0.3956318      1.4529828   1.0000000    1.41D-08
      < 2p| 3p>= 6.0D-09
                     2p     15.9592124      4.7810460   1.0000000**  9.70D-09
      < 1s| 3s>=-4.4D-07
      < 2s| 3s>=-5.6D-08
                     3s     16.2077493     14.8282834   1.0000000**  5.32D-07
      < 2p| 3p>=-6.7D-09
                     3p     30.8279905     66.8115267   1.0000000**  2.45D-08
                     3d     35.1829408     88.8117857   1.0000000**  2.94D-10

         ETOTAL=    -7.47318427   Loops,DeltaE,Res.:   2  3.955D-16  7.343D-09
      1  0.9984151      2  0.0004030      3  0.0002912      4 -0.0024959
      5  0.0016210      6 -0.0009009      7  0.0029624      8 -0.0015623

         Sum of ETOTAL :      -7.47318427
  DeltaE =  1.77635683940025046E-015 Sum_Energy =   -7.4731842652568732     


     ENERGY (a.u.)
     ------
           Total                  -7.473184265
           Potential             -14.946368479
           Kinetic                 7.473184213
           Ratio                   2.000000007

*******************************************************************************
*         SAVE OUTPUT FILES                                                   * 
*******************************************************************************

>>cp wfn.out 2SeCAS3.w
>>cp cfg.inp 2SeCAS3.c
>>cp 2Se.l 2SeCAS3.l
>>cp summry 2SeCAS3.s

*******************************************************************************
*         RUN HFS TO COMPUTE HYPERFINE INTERACTION CONSTANTS                  *
*         INPUT FILES: 2SeCAS3.c, 2SeCAS3.w, 2SeCAS3.l                        *
*         OUTPUT FILES: 2SeCAS3.h                                             *  
*                                                                             *
*         NOTE1: THE OPTION MCHF IS OBSOLETE AND ADERES TO THE OLD OUTPUT     *
*         FORMAT OF MCHF. NOW YOU NEED ALWAYS ASK FOR INPUT FROM A CI         *
*         CALCULATION.                                                        *
*                                                                             *
*         NOTE2: THE HFS PROGRAM ONLY COMPUTES THE HYPERFINE INTERACTION      *
*         FROM ONE STATE AND YOU HAVE TO SPECIFY THE POSITION OF THE LEADING  *
*         CSF OF THIS STATE. AT SOME POINT SOMEONE SHOULD RECODE THE MODULE   * 
*******************************************************************************

 Name of state ...
>>2SeCAS3

                     Hyperfine structure calculation

 Electron density at the nucleus ? (Y/N)
>>y
 Indicate the type of calculation
 0 => diagonal A and B factors only;
 1 => diagonal and off-diagonal A and B factors;
>>0
 Input from an MCHF (M) or CI (C) calculation ?
>>C
 Is the CI calculation J dependant ? (Y/N)
>>n
 Give the index of the dominant cfg in the CI
 expansion for which the hfs is to be calculated ?
>>1
 Give 2*I and nuclear dipole and quadrupole moments (in n.m. and barns)
>>2,1,1

                     The configuration set


 STATE  (WITH      27 CONGIGURATIONS):
 ------------------------------------


 THERE ARE  6 ORBITALS AS FOLLOWS:

       1s  2s  2p  3s  3p  3d

 THERE ARE  0 CLOSED SUBSHELLS COMMON TO ALL CONFIGURATIONS AS FOLLOWS:


 ......................

    ja =          10
    ja =          20
 PER TEST


 END OF CASE
===========

 Total CPU time was    0.000 minutes



*******************************************************************************
*         RUN LSGEN TO GENERATE N = 3 CAS BREIT CONFIGURATION LIST FOR 2S     *
*         OUTPUT FILES: clist.out, clist.log                                  *
*******************************************************************************

 New list, add to existing list, expand existing list, optimized sorting, restored order or quit? (*/a/e/s/r/q)
>>
 Breit or MCHF? (B/*)
>>B
 Default, symmetry or user specified ordering? (*/s/u)
>>
 Highest principal quantum number, n? (1..15)
>>3
 Highest orbital angular momentum, l? (s..d)
>>d
 Are all these nl-subshells active? (n/*)
>>
 Limitations on population of n-subshells? (y/*)
>>
 Highest n-number in reference configuration? (1..3)
>>2
 Number of electrons in 1s? (0..2)
>>2
 Number of electrons in 2s? (0..2)
>>1
 Number of electrons in 2p? (0..6)
>>0
 Maximum 2*J-value? (0..)
>>1
 Minimum 2*J-value? (0..1)
>>1
 Maximum (2*S+1)-value? (1..9)
>>9
 Minimum (2*S+1)-value? (1..9)
>>1
 Maximum resulting angular momentum? (S..N/N=*)
>>
 Minimum resulting angular momentum? (S..N/S=*)
>>
 Number of excitations = ? (0..3)
>>3
 79 configuration states have been generated.
 Generate a second list? (y/*)
>>
 79 configuration states in the final list.
 The generated file is called clist.out.
FORTRAN STOP

*******************************************************************************
*         COPY FILES                                                          *
*         IT IS ADVISABLE TO SAVE THE LSGEN LOG-FILE TO HAVE A RECORD ON      *
*         HOW THE CONFIGURATION LISTS GENERATION WAS DONE                     *
*******************************************************************************

>>cp clist.log 2SeCAS3BREIT.log
>>cp clist.out 2SeCAS3BREIT.c

*******************************************************************************
*         COPY WAVE FUNCTION FILES TO PREPARE FOR THE BREIT-PAULI RUN         *
*******************************************************************************

>>cp 2SeCAS3.w 2SeCAS3BREIT.w

*******************************************************************************
*         RUN CI BREIT-PAULI                                                  *
*         INPUT FILES: 2SeCAS3BREIT.c, 2SeCAS3BREIT.w                         *
*         OUTPUT FILES: 2SeCASBREIT.j                                         *
*******************************************************************************

>>bpci
 Enter ATOM, relativistic (Y/N) with mass correction (Y/N)
>>2SeCAS3BREIT,y,n
 Restarting (Y/y) ?
>>n
 Use existing Matrix and <atom>.l/j initial guess (Y/y)?
>>n

 Enter Maximum and minimum values of 2*J
>>1,1

 Enter eigenvalues: one line per term, eigenvalues separated by commas
 2*J =  1
>>1

                    =======================
                     B R E I T - P A U L I
                    =======================

 Indicate the type of calculation
 0 => non-relativistic Hamiltonian only;
 1 => one or more relativistic operators only;
 2 => non-relativistic operators and selected relativistic:
>>2
 All relativistic operators ? (Y/N)
>>y


 THE TYPE OF CALCULATION IS DEFINED BY THE FOLLOWING PARAMETERS -
      BREIT-PAULI OPERATORS             IREL   = 2
      PHASE CONVENTION PARAMETER        ICSTAS = 1

                             ---------------------
                             THE CONFIGURATION SET
                             ---------------------





 STATE  (WITH      79 CONGIGURATIONS):
 ------------------------------------


 THERE ARE  6 ORBITALS AS FOLLOWS:

       1s  2s  2p  3s  3p  3d

 THERE ARE  0 CLOSED SUBSHELLS COMMON TO ALL CONFIGURATIONS AS FOLLOWS:


 All Interactions? (Y/N):
>>y

 Default Rydberg constant (y/n)
>>y

......

 Allocating space for           783  integrals
 Alcmat allocations for idisk=            0 nze =          659
 J =            1            1
 Entering LSJMAT with 2J =            1  NUME =            1
 LSJMAT with idisk=            0 Nze =         2636

 Summary of Davidson Performance
 ===============================
 Number of Iterations:            10
 Shifted Eigval's:  -0.2830587230149465
 Delta Lambda:       3.3306690738754696E-016
 Residuals:          7.8543171982727995E-009


     1 Eigenvalues found
 Finished with Davidson
 Leaving LSJMAT
 onlydvd  F
 ILS =            0
 Finished with the file


*******************************************************************************
*          RUN HF FOR 1s(2)2p 2P                                              *
*          OUTPUT FILE: wfn.out, hf.log                                       *
*******************************************************************************

>>HF

                      =============================
                       H A R T R E E - F O C K . 96
                      =============================


               THE DIMENSIONS FOR THE CURRENT VERSION ARE:
                          NWF= 20        NO=220



 START OF CASE
 =============


 Enter ATOM,TERM,Z
 Examples: O,3P,8. or Oxygen,AV,8.
>>Li,2P,3.

 List the CLOSED shells in the fields indicated (blank line if none)
 ... ... ... ... ... ... ... ... etc.
>>1s

 Enter electrons outside CLOSED shells (blank line if none)
 Example: 2s(1)2p(3)
>>2p(1)

 There are   2 orbitals as follows:
   1s  2p

 Orbitals to be varied: ALL/NONE/=i (last i)/comma delimited list/H
>>all

 Default electron parameters ? (Y/N/H)
>>y

 Default values for remaining parameters? (Y/N/H)
>>y
 
 
 ..................

         ITERATION NUMBER  5
          ----------------

          SCF CONVERGENCE CRITERIA (SCFTOL*SQRT(Z*NWF)) =   3.9D-07


                     EL         ED             AZ           NORM       DPM
                     1s      5.0614517      9.2618326   0.9999999    2.29D-08
                     2p      0.2573450      0.4261374   1.0000000    9.86D-09




     TOTAL ENERGY (a.u.)
     ----- ------
           Non-Relativistic       -7.36506966    Kinetic        7.36506968
           Relativistic Shift     -0.00053279    Potential    -14.73013934
           Relativistic           -7.36560245    Ratio        -1.999999997
 
 Additional parameters ? (Y/N/H)
>>n

 Do you wish to continue along the sequence ?
>>n


 END OF CASE
 ===========
 
*******************************************************************************
*         COPY FILES                                                          *
*         Keep a copy of the wfn.out in 2PoDF.w file for reference            *
*         Keep a copy of hf.log in 2PoDF.log                                  *
*******************************************************************************

cp wfn.out wfn.inp
cp wfn.out 2PoDF.w
cp hf.log 2PoDF.log

*******************************************************************************
*         RUN LSGEN TO GENERATE N = 3 CAS CONFIGURATION LIST FOR 2P           *
*         OUTPUT FILES: clist.out, clist.log                                  *
*******************************************************************************

 New list, add to existing list, expand existing list, optimized sorting, restored order or quit? (*/a/e/s/r/q)
>>
 Breit or MCHF? (B/*)
>>
 Default, symmetry or user specified ordering? (*/s/u)
>>
 Highest principal quantum number, n? (1..15)
>>3
 Highest orbital angular momentum, l? (s..d)
>>d
 Are all these nl-subshells active? (n/*)
>>
 Limitations on population of n-subshells? (y/*)
>>
 Highest n-number in reference configuration? (1..3)
>>2
 Number of electrons in 1s? (0..2)
>>2
 Number of electrons in 2s? (0..2)
>>0
 Number of electrons in 2p? (0..6)
>>1
 Resulting term? (1S, 3P, etc.)
>>2P
 Number of excitations = ? (0..3)
>>3
 44 configuration states have been generated.
 Generate a second list? (y/*)
>>
 44 configuration states in the final list.
 The generated file is called clist.out.
FORTRAN STOP

*******************************************************************************
*         COPY FILES                                                          *
*         IT IS ADVISABLE TO SAVE THE LSGEN LOG-FILE TO HAVE A RECORD ON      *
*         HOW THE CONFIGURATION LISTS GENERATION WAS DONE                     *
*******************************************************************************

>>cp clist.log 2PoCAS3.log
>>cp clist.out cfg.inp

*******************************************************************************
*         RUN NONH TO GENERATE ENERGY EXPRESSION                              *
*         INPUT FILES: cfg.inp                                                *
*         OUTPUT FILES: cfg.h, yint.lst, c.lst, ih.=n.lst                     *
*******************************************************************************

>>nonh

 input file is cfg.inp ...



                       ===============================
                             N O N H       2000
                       ===============================



 THERE ARE  6 ORBITALS AS FOLLOWS:

       1s  2s  2p  3s  3p  3d

 THERE ARE  0 CLOSED SUBSHELLS COMMON TO ALL CONFIGURATIONS AS FOLLOWS:


 Allocating space for           214  integrals
 processing 2Po with           44 configurations
     649 non-zero matrix elements
     152 NF     166 NG    1083 NR     191 NL
    1592 Total number of integrals

 end-of-file clist!!!

*******************************************************************************
*         RUN MCHF TO OBTAIN SELF CONSISTENT SOLUTIONS                        *
*         INPUT FILES: wfn.inp (optional), cfg.inp, angular files             *
*         OUTPUT FILES: wfn.out, 2Po.l, summry                                *
*                                                                             *
*         NOTE1: Mixing coefficients written to a file named by LS-symmetry   *
*         and parity, in this case 2Po.l                                      *
*                                                                             *
*         NOTE2: We force the program to iterate to higher precision than     *
*         default which is a good thing for small calculations                *
*******************************************************************************

>>mchf

                    =======================
                            M C H F  ... 2000
                    =======================




          THE DIMENSIONS FOR THE CURRENT VERSION ARE:
             NWD= 60        NO=220    Lagrange Multipliers=800



 START OF CASE
 =============


 ATOM, Z in FORMAT(A, F) :
>>Li,3.
 cfg.inp has configurations for             1  terms

 Enter eigenvalues and weights: one line per term, eigenvalues with weights
 in parenthesis and separated by commas, default is 1.0
 2Po
>>1

 There are   6 orbitals as follows:
   1s  2s  2p  3s  3p  3d
 Enter orbitals to be varied: (ALL,NONE,SOME,NIT=,comma delimited list)
>>all
 Enter those that are spectroscopic
>>1s,2p

 Default electron parameters ? (Y/N)
>>y

 Default values for other parameters? (Y/N)
>>n

 Default values (NO,REL,STRONG) ? (Y/N)
>>y

 Li        3.   220  6  6  0  F

 Default values for PRINT, CFGTOL, SCFTOL ? (Y/N)
>>n
 Input free FORMAT(L, F, F)
>>.f.,1.e-50,1.e-50

 Default values for NSCF, IC ? (Y/N)
>>n
 Input free FORMAT(I, I)
>>300,0

 Default values for ACFG,LD,TRACE? (Y/N)
>>y

 .................
 
          ITERATION NUMBER  62
          ----------------

          CONVERGENCE CRITERIA:ENERGY  (CFGTOL)            =  1.0D-50
                              :FUNCTION(SCFTOL*SQRT(Z*NWF))=  4.2D-50

          E( 2s 1s) =    25.62110   E( 1s 2s) =     0.03566
          E( 3s 1s) =   125.27085   E( 1s 3s) =     0.00805
          E( 3s 2s) =    72.76330   E( 2s 3s) =     3.36051
          E( 3p 2p) =     2.18254   E( 2p 3p) =     0.00707

                     EL         ED             AZ           NORM       DPM
      < 1s| 2s>= 9.9D-10
      < 1s| 3s>=-5.8D-11
                     1s      5.1380025      9.2694030   1.0000000    1.23D-09
      < 2s| 3s>=-2.8D-07
      < 1s| 2s>=-3.6D-07
                     2s     15.1297623      9.8051412   1.0000000**  8.16D-07
      < 2p| 3p>=-7.7D-11
                     2p      0.2587501      0.4284629   1.0000000    1.50D-10
      < 1s| 3s>= 4.6D-08
      < 2s| 3s>=-3.2D-07
                     3s     32.0103498     25.1021580   1.0000000**  3.78D-07
      < 2p| 3p>=-2.0D-10
                     3p     17.5853816     23.8010905   1.0000000**  2.18D-09
                     3d     32.5391413     81.3500754   1.0000000**  1.10D-09

         ETOTAL=    -7.40458774   Loops,DeltaE,Res.:   2  3.608D-15  5.517D-08
      1  0.9984119      2  0.0003758      3 -0.0006572      4 -0.0014719
      5 -0.0042385      6 -0.0027196      7 -0.0003308      8 -0.0001325

         Sum of ETOTAL :      -7.40458774
  DeltaE =  6.21724893790087663E-015 Sum_Energy =   -7.4045877355330409     


     ENERGY (a.u.)
     ------
           Total                  -7.404587736
           Potential             -14.809175438
           Kinetic                 7.404587702
           Ratio                   2.000000005

*******************************************************************************
*         SAVE OUTPUT FILES                                                   * 
*******************************************************************************

>>cp wfn.out 2PoCAS3.w
>>cp cfg.inp 2PoCAS3.c
>>cp 2Po.l 2PoCAS3.l
>>cp summry 2PoCAS3.s

*******************************************************************************
*         RUN HFS TO COMPUTE HYPERFINE INTERACTION CONSTANTS                  *
*         INPUT FILES: 2PoCAS3.c, 2PoCAS3.w, 2PoCAS3.l                        *
*         OUTPUT FILES: 2PoCAS3.h                                             *  
*                                                                             *
*         NOTE1: THE OPTION MCHF IS OBSOLETE AND ADERES TO THE OLD OUTPUT     *
*         FORMAT OF MCHF. NOW YOU NEED ALWAYS ASK FOR INPUT FROM A CI         *
*         CALCULATION.                                                        *
*                                                                             *
*         NOTE2: THE HFS PROGRAM ONLY COMPUTES THE HYPERFINE INTERACTION      *
*         FROM ONE STATE AND YOU HAVE TO SPECIFY THE POSITION OF THE LEADING  *
*         CSF OF THIS STATE. AT SOME POINT SOMEONE SHOULD RECODE THE MODULE   * 
*******************************************************************************

 Name of state ...
>>2PoCAS3

                     Hyperfine structure calculation

 Electron density at the nucleus ? (Y/N)
>>y
 Indicate the type of calculation
 0 => diagonal A and B factors only;
 1 => diagonal and off-diagonal A and B factors;
>>0
 Input from an MCHF (M) or CI (C) calculation ?
>>C
 Is the CI calculation J dependant ? (Y/N)
>>n
 Give the index of the dominant cfg in the CI
 expansion for which the hfs is to be calculated ?
>>1
 Give 2*I and nuclear dipole and quadrupole moments (in n.m. and barns)
>>2,1,1

                     The configuration set


 STATE  (WITH      44 CONGIGURATIONS):
 ------------------------------------


 THERE ARE  6 ORBITALS AS FOLLOWS:

       1s  2s  2p  3s  3p  3d

 THERE ARE  0 CLOSED SUBSHELLS COMMON TO ALL CONFIGURATIONS AS FOLLOWS:


 ......................

    ja =          10
    ja =          20
    ja =          30
    ja =          40
 PER TEST


 END OF CASE
===========

 Total CPU time was    0.000 minutes



*******************************************************************************
*         RUN LSGEN TO GENERATE N = 3 CAS BREIT CONFIGURATION LIST FOR 2P     *
*         OUTPUT FILES: clist.out, clist.log                                  *
*******************************************************************************

 New list, add to existing list, expand existing list, optimized sorting, restored order or quit? (*/a/e/s/r/q)
>>
 Breit or MCHF? (B/*)
>>B
 Default, symmetry or user specified ordering? (*/s/u)
>>
 Highest principal quantum number, n? (1..15)
>>3
 Highest orbital angular momentum, l? (s..d)
>>d
 Are all these nl-subshells active? (n/*)
>>
 Limitations on population of n-subshells? (y/*)
>>
 Highest n-number in reference configuration? (1..3)
>>2
 Number of electrons in 1s? (0..2)
>>2
 Number of electrons in 2s? (0..2)
>>0
 Number of electrons in 2p? (0..6)
>>1
 Maximum 2*J-value? (0..)
>>3
 Minimum 2*J-value? (0..1)
>>1
 Maximum (2*S+1)-value? (1..9)
>>9
 Minimum (2*S+1)-value? (1..9)
>>1
 Maximum resulting angular momentum? (S..N/N=*)
>>
 Minimum resulting angular momentum? (S..N/S=*)
>>
 Number of excitations = ? (0..3)
>>3
 114 configuration states have been generated.
 Generate a second list? (y/*)
>>
 114 configuration states in the final list.
 The generated file is called clist.out.
FORTRAN STOP

*******************************************************************************
*         COPY FILES                                                          *
*         IT IS ADVISABLE TO SAVE THE LSGEN LOG-FILE TO HAVE A RECORD ON      *
*         HOW THE CONFIGURATION LISTS GENERATION WAS DONE                     *
*******************************************************************************

>>cp clist.log 2PoCAS3BREIT.log
>>cp clist.out 2PoCAS3BREIT.c

*******************************************************************************
*         COPY WAVE FUNCTION FILES TO PREPARE FOR THE BREIT-PAULI RUN         *
*******************************************************************************

>>cp 2PoCAS3.w 2PoCAS3BREIT.w

*******************************************************************************
*         RUN CI BREIT-PAULI                                                  *
*         INPUT FILES: 2PoCAS3BREIT.c, 2PoCAS3BREIT.w                         *
*         OUTPUT FILES: 2PoCASBREIT.j                                         *
*******************************************************************************

>>bpci
 Enter ATOM, relativistic (Y/N) with mass correction (Y/N)
>>2PoCAS3BREIT,y,n
 Restarting (Y/y) ?
>>n
 Use existing Matrix and <atom>.l/j initial guess (Y/y)?
>>n

 Enter Maximum and minimum values of 2*J
>>3,1

 Enter eigenvalues: one line per term, eigenvalues separated by commas
 2*J =  3
>>1
 Enter eigenvalues: one line per term, eigenvalues separated by commas
 2*J =  1
>>1

                    =======================
                     B R E I T - P A U L I
                    =======================

 Indicate the type of calculation
 0 => non-relativistic Hamiltonian only;
 1 => one or more relativistic operators only;
 2 => non-relativistic operators and selected relativistic:
>>2
 All relativistic operators ? (Y/N)
>>y


 THE TYPE OF CALCULATION IS DEFINED BY THE FOLLOWING PARAMETERS -
      BREIT-PAULI OPERATORS             IREL   = 2
      PHASE CONVENTION PARAMETER        ICSTAS = 1

                             ---------------------
                             THE CONFIGURATION SET
                             ---------------------





 STATE  (WITH      114 CONGIGURATIONS):
 ------------------------------------


 THERE ARE  6 ORBITALS AS FOLLOWS:

       1s  2s  2p  3s  3p  3d

 THERE ARE  0 CLOSED SUBSHELLS COMMON TO ALL CONFIGURATIONS AS FOLLOWS:


 All Interactions? (Y/N):
>>y

 Default Rydberg constant (y/n)
>>y

......

 Allocating space for          783  integrals
   jb =      100
 Alcmat allocations for idisk=           0 nze =        1124
 J =           3           1
 Entering LSJMAT with 2J =           3  NUME =           1
 LSJMAT with idisk=           0 Nze =        8992

 Summary of Davidson Performance
 ===============================
 Number of Iterations:            9
 Shifted Eigval's: -3.95974915438215003E-002
 Delta Lambda:      6.10622663543836097E-016
 Residuals:         1.09932870175402541E-008


     1 Eigenvalues found
 Finished with Davidson
 Leaving LSJMAT
 J =           1           1
 Entering LSJMAT with 2J =           1  NUME =           1
 LSJMAT with idisk=           0 Nze =        8992

 Summary of Davidson Performance
 ===============================
 Number of Iterations:            9
 Shifted Eigval's: -3.95976088106196750E-002
 Delta Lambda:      3.33066907387546962E-016
 Residuals:         1.10031822657079236E-008


     1 Eigenvalues found
 Finished with Davidson
 Leaving LSJMAT
 onlydvd F
 ILS =           0
 Finished with the file
 
*******************************************************************************
*         RUN BIOTR TO COMPUTE TRANSITION RATES                               *
*         INPUT FILES: 2SoCAS3BREIT.c, 2SoCAS3BREIT.w, 2SoCAS3BREIT.j         *
*                      2PoCAS3BREIT.c, 2PoCAS3BREIT.w, 2PoCAS3BREIT.j         *
*         OUTPUT FILES: 2SoCAS3BREIT.2PoCASBREIT.lsj                          *
******************************************************************************* 
 
 
>>biotr

                    ========================
                      T R A N S B I O  99 
                    ========================


  Name of Initial State
>>2SeCAS3BREIT
  Name of Final State
>>2PoCAS3BREIT
  intermediate printing (y or n) ?  
>>n
  Relativistic calculation ? (y/n) 
>>y
  Type of transition ? (E1, E2, M1, M2, .. or *) 
>>E1
 
 .............................
 
------------------------------------------------------
Pair number   1



 Initial CSF : 1s(2).2s_2S                                        J = 0.5
 Final   CSF : 1s(2).2p_2P                                        J = 1.5

 2*j =     1 lbl =     1 total energy =       -7.4738052
 2*j =     3 lbl =     1 total energy =       -7.4051932


          LENGTH   FORMALISM: 
          -------- ----------


          SL                                       =   2.2537890D+01
          FINAL OSCILLATOR STRENGTH (GF)           =   1.0308323D+00
          TRANSITION PROBABILITY IN EMISSION (Aki) =   3.8973684D+07




          VELOCITY FORMALISM: 
          -------- ----------


          SV                                       =   2.1851565D+01
          FINAL OSCILLATOR STRENGTH (GF)           =   9.9944136D-01
          TRANSITION PROBABILITY IN EMISSION (Aki) =   3.7786856D+07


  npair =            2

------------------------------------------------------
Pair number   2



 Initial CSF : 1s(2).2s_2S                                        J = 0.5
 Final   CSF : 1s(2).2p_2P                                        J = 0.5

 2*j =     1 lbl =     1 total energy =       -7.4738052
 2*j =     1 lbl =     1 total energy =       -7.4051945


          LENGTH   FORMALISM: 
          -------- ----------


          SL                                       =   1.1268927D+01
          FINAL OSCILLATOR STRENGTH (GF)           =   5.1540601D-01
          TRANSITION PROBABILITY IN EMISSION (Aki) =   3.8971507D+07




          VELOCITY FORMALISM: 
          -------- ----------


          SV                                       =   1.0925995D+01
          FINAL OSCILLATOR STRENGTH (GF)           =   4.9972136D-01
          TRANSITION PROBABILITY IN EMISSION (Aki) =   3.7785540D+07


  Type of transition ? (E1, E2, M1, M2, .. or *) 
*
STOP  END OF CASE
 

\end{verbatim}     
\chapter{Spectrum calculations}
In practical applications the wave functions for many states with different LS symmetries are determined at the same time. 
\section{First example: states of $2s^22p^2$ and $2s2p^ 3$ in O III}
Below is the NIST table for O III

\begin{verbatim}
---------------------------------------------------------------------------
Configuration          | Term    |   J |              Level    | Reference
-----------------------|---------|-----|-----------------------|-----------
                       |         |     |                       |           
2s2.2p2                | 3P      |   0 |              0.000    |     L7288
                       |         |   1 |            113.178    |          
                       |         |   2 |            306.174    |          
                       |         |     |                       |           
2s2.2p2                | 1D      |   2 |          20273.27     |          
                       |         |     |                       |           
2s2.2p2                | 1S      |   0 |          43185.74     |          
                       |         |     |                       |           
2s.2p3                 | 5S*     |   2 |          60324.79     |          
                       |         |     |                       |           
2s.2p3                 | 3D*     |   3 |         120025.2      |          
                       |         |   2 |         120053.4      |          
                       |         |   1 |         120058.2      |          
                       |         |     |                       |           
2s.2p3                 | 3P*     |   2 |         142381.0      |          
                       |         |   1 |         142381.8      |          
                       |         |   0 |         142393.5      |          
                       |         |     |                       |           
2s.2p3                 | 1D*     |   2 |         187054.0      |          
                       |         |     |                       |           
2s.2p3                 | 3S*     |   1 |         197087.7      |          
                       |         |     |                       |           
2s.2p3                 | 1P*     |   1 |         210461.8      |          
                       |         |     |                       |           
2s2.2p.(2P*).3s        | 3P*     |   0 |         267258.71     |          
                       |         |   1 |         267377.11     |          
                       |         |   2 |         267634.00     |          
                       |         |     |                       |           
2s2.2p.(2P*).3s        | 1P*     |   1 |         273081.33     |          
                       |         |     |                       |           
2p4                    | 3P      |   2 |         283759.70     |          
                       |         |   1 |         283977.40     |          
                       |         |   0 |         284071.90     |      
\end{verbatim}
We want to compute wave functions for all states belonging to $2s^22p^2$ and all states belonging to $2s2p^3$.
As a starting point we generate, in one calculation, non-relativistic wave functions for $2s^22p^2\{^3P, {^1D}, {^1S}\}$ and, in another calculation, non-relativistic wave functions for
$2s2p^3\{^3D,{^3P},{^1D},{^3S},{^1P}\}$. The correlation model is single (S) and (D) excitations from the 
$\{1s^22s^22p^2,1s^22p^4\}$ multireference (MR) to an orbital set with $n=3$ and  (S) and (D) excitations from the 
$1s^22s2p^3$ reference to an orbital set with $n=3$. 
Then relativistic effects are taken into account in Breit-Pauli.                       
\subsection*{Overview}
\begin{enumerate}
\item Perform HF calculation for the average of $1s^22s^22p^2$.
\item Generate $n = 3$ SD-MR configuration list using \verb+csfexcitation+ 
for each of the $\{^3P,{^1D},{^1S}\}$ terms of $1s^22s^22p^2$. Concatenate to one list
\item Perform angular integration.
\item Perform self-consistent field calculation optimizing on states of several LS terms
\item Save output to \verb+even3+. 
\item Generate $n = 3$ SD-MR Breit Pauli configuration list  
\item Perform CI Breit-Pauli calculation 
\item Perform HF calculation for the average of $1s^22s2p^3$.
\item Generate $n = 3$ SD-MR configuration list using \verb+csfexcitation+ 
for each of the $\{{^3D},{^3P},{^1D},{^3S},{^1P}\}$ terms of $1s^22s2p^3$. Concatenate to one list
\item Perform angular integration.
\item Perform self-consistent field calculation optimizing on states of several LS terms
\item Save output to \verb+odd3+. 
\item Generate $n = 3$ SD-MR Breit Pauli configuration list   
\item Perform CI Breit-Pauli calculation 
\item Compute the transition rates from the CI wave functions.
\end{enumerate} 

\subsection*{Program input}
In the test-runs input is marked by \verb+>>+ and \verb+>>3+, for example, indicate that the user should input 3 and then strike the return key.
When \verb+>>+ is followed by blanks just strike the return key.
\begin{verbatim}

*******************************************************************************
*          RUN HF FOR AVERAGE of 1s(2)2s(2)2p(2)                              *
*          OUTPUT FILE: wfn.out, hf.log                                       *
*******************************************************************************

>>HF

                      =============================
                       H A R T R E E - F O C K . 86
                      =============================


               THE DIMENSIONS FOR THE CURRENT VERSION ARE:
                          NWF= 20        NO=220



 START OF CASE
 =============


 Enter ATOM,TERM,Z
 Examples: O,3P,8. or Oxygen,AV,8.
>>O,AV,8.

 List the CLOSED shells in the fields indicated (blank line if none)
 ... ... ... ... ... ... ... ... etc.
>>1s  2s

 Enter electrons outside CLOSED shells (blank line if none)
 Example: 2s(1)2p(3)
>>2p(2)

 There are   3 orbitals as follows:
   1s  2s  2p

 Orbitals to be varied: ALL/NONE/=i (last i)/comma delimited list/H
>>all

 Default electron parameters ? (Y/N/H) 
>>y
          1s     1.00     44.538    43.351    SCALED O     AV_E      
          2s     3.00      5.081    10.173    SCALED O     AV_E      
          2p     4.50      3.925    17.791    SCALED O     AV_E      

 Default values for remaining parameters? (Y/N/H) 
>>y

.....................


          ITERATION NUMBER  5
          ----------------

          SCF CONVERGENCE CRITERIA (SCFTOL*SQRT(Z*NWF)) =   7.8D-07


                     EL         ED             AZ           NORM       DPM
                     1s     44.5448235     43.1949784   1.0000000    9.70D-09
                     2s      4.9687528     10.7634474   1.0000000    4.54D-08
                     2p      3.9889024     17.9466483   1.0000000    1.23D-08


      < 1s| 2s>= 4.9D-09


     TOTAL ENERGY (a.u.)
     ----- ------
           Non-Relativistic      -73.04850746    Kinetic       73.04850725
           Relativistic Shift     -0.04960044    Potential   -146.09701471
           Relativistic          -73.09810790    Ratio        -2.000000003


 Additional parameters ? (Y/N/H)
>>n

 Do you wish to continue along the sequence ?
>>n


 END OF CASE
 ===========
 
*******************************************************************************
*         COPY FILES                                                          *
*         Keep a copy of the wfn.out in evenDF.w file for reference           *
*         Keep a copy of hf.log in evenDF.log                                 *
*******************************************************************************

cp wfn.out wfn.inp
cp wfn.out evenDF.w
cp hf.log evenDF.log

*******************************************************************************
*         RUN CSFEXCITATIONS TO GENERATE SD-MR EXPANSIONS FOR EACH LS TERM    *
*         AT THE END CONCATENATE FILES TO cfg.inp                             *
*******************************************************************************

*******************************************************************************
*         RUN FOR 3P                                                          *
*******************************************************************************

>>csfexcitation

 CSFEXCITATION
 This program creates excitation input to LSGEN
 Configurations should be entered in spectroscopic notation
 with occupation numbers and indications if orbitals are
 closed (c), inactive (i), active (*) or has a minimal
 occupation e.g. 1s(2,1)2s(2,*)
 Outputfiles: excitationdata.sh, csfexcitation.log

 Enter list of (maximum 100) configurations. End list with a blank line or an astersik (*).

 Give configuration           1
>>1s(2,*)2s(2,*)2p(2,*)
 Give configuration           2
>>1s(2,*)2p(4,*)
 Give configuration           3
>>
 Give set of active orbitals in a comma delimited list ordered by l-symmetry, e.g. 5s,4p,3d
>>3s,3p,3d
 Resulting term? (1S, 3P, etc.)
>>3P
 Number of excitations (if negative number e.g. -2, correlation 
 orbitals will always be doubly occupied)                        
>>2
 Generate more lists ? (y/n)
>>n

*******************************************************************************
*         RUN THE SHELL SCRIPT excitationdata.sh PRODUCED BY CSFEXCITATION    *
*******************************************************************************

>>./excitationdata.sh

..........

 173 configuration states in the final list.
 The merged file is called clist.out.

*******************************************************************************
*         SAVE IN even33P                                                     *
*******************************************************************************

>>mv clist.out even33P

*******************************************************************************
*         RUN FOR 1D                                                          *
*******************************************************************************

>>csfexcitation

 CSFEXCITATION
 This program creates excitation input to LSGEN
 Configurations should be entered in spectroscopic notation
 with occupation numbers and indications if orbitals are
 closed (c), inactive (i), active (*) or has a minimal
 occupation e.g. 1s(2,1)2s(2,*)
 Outputfiles: excitationdata.sh, csfexcitation.log

 Enter list of (maximum 100) configurations. End list with a blank line or an astersik (*).

 Give configuration           1
>>1s(2,*)2s(2,*)2p(2,*)
 Give configuration           2
>>1s(2,*)2p(4,*)
 Give configuration           3
>>
 Give set of active orbitals in a comma delimited list ordered by l-symmetry, e.g. 5s,4p,3d
>>3s,3p,3d
 Resulting term? (1S, 3P, etc.)
>>1D
 Number of excitations (if negative number e.g. -2, correlation 
 orbitals will always be doubly occupied)                        
>>2
 Generate more lists ? (y/n)
>>n

*******************************************************************************
*         RUN THE SHELL SCRIPT excitationdata.sh PRODUCED BY CSFEXCITATION    *
*******************************************************************************

>>./excitationdata.sh

..........

 134 configuration states in the final list.
 The merged file is called clist.out.

*******************************************************************************
*         SAVE IN even31D                                                     *
*******************************************************************************

>>mv clist.out even31D

*******************************************************************************
*         RUN FOR 1S                                                          *
*******************************************************************************

>>csfexcitation

 CSFEXCITATION
 This program creates excitation input to LSGEN
 Configurations should be entered in spectroscopic notation
 with occupation numbers and indications if orbitals are
 closed (c), inactive (i), active (*) or has a minimal
 occupation e.g. 1s(2,1)2s(2,*)
 Outputfiles: excitationdata.sh, csfexcitation.log

 Enter list of (maximum 100) configurations. End list with a blank line or an astersik (*).

 Give configuration           1
>>1s(2,*)2s(2,*)2p(2,*)
 Give configuration           2
>>1s(2,*)2p(4,*)
 Give configuration           3
>>
 Give set of active orbitals in a comma delimited list ordered by l-symmetry, e.g. 5s,4p,3d
>>3s,3p,3d
 Resulting term? (1S, 3P, etc.)
>>1S
 Number of excitations (if negative number e.g. -2, correlation 
 orbitals will always be doubly occupied)                        
>>2
 Generate more lists ? (y/n)
>>n

*******************************************************************************
*         RUN THE SHELL SCRIPT excitationdata.sh PRODUCED BY CSFEXCITATION    *
*******************************************************************************

>>./excitationdata.sh

..........

 71 configuration states in the final list.
 The merged file is called clist.out.

*******************************************************************************
*         SAVE IN even31S                                                     *
*******************************************************************************

>>mv clist.out even31S

*******************************************************************************
*         CONCATENATE THE FILES                                               *
*******************************************************************************

>>cat even33P even31D even31S > cfg.inp

*******************************************************************************
*         RUN NONH TO GENERATE ENERGY EXPRESSION                              *
*         INPUT FILES: cfg.inp                                                *
*         OUTPUT FILES: cfg.h, yint.lst, c.lst, ih.=n.lst                     *
*******************************************************************************

>>nonh

                       ===============================
                             N O N H       2000
                       ===============================


 input file is cfg.inp ...


 THERE ARE  6 ORBITALS AS FOLLOWS:

       1s  2s  2p  3s  3p  3d

 THERE ARE  0 CLOSED SUBSHELLS COMMON TO ALL CONFIGURATIONS AS FOLLOWS:


 Allocating space for          214  integrals
 processing 3Pe with          173 configurations
    jb =         100
    4945 non-zero matrix elements
    1673 NF    2214 NG    7780 NR     923 NL
   12590 Total number of integrals
 processing 1De with          134 configurations
    jb =         100
    2865 non-zero matrix elements
    1309 NF    1417 NG    4700 NR     664 NL
    8090 Total number of integrals
 processing 1Se with           71 configurations
     966 non-zero matrix elements
     612 NF     499 NG    1682 NR     336 NL
    3129 Total number of integrals

 end-of-file clist!!!


*******************************************************************************
*         RUN MCHF TO OBTAIN SELF CONSISTENT SOLUTIONS                        *
*         INPUT FILES: wfn.inp (optional), cfg.inp, angular files             *
*         OUTPUT FILES: wfn.out, 3Pe.l, 1De.l 1Se.l summry                    *
*                                                                             *
*         NOTE1: Mixing coefficients written to a files named by LS-symmetry  *
*         and parity                                                          *
*                                                                             *
*         NOTE2: We force the program to iterate to higher precision than     *
*         default which is a good thing for small calculations                *
*******************************************************************************

>>mchf

                    =======================
                            M C H F  ... 2000   
                    =======================




          THE DIMENSIONS FOR THE CURRENT VERSION ARE:
             NWD= 60        NO=220    Lagrange Multipliers=800



 START OF CASE
 =============


 ATOM, Z in FORMAT(A, F) : 
>>O,8.
 cfg.inp has configurations for            3  terms

 Enter eigenvalues and weights: one line per term, eigenvalues with weights
 in parenthesis and separated by commas, default is 1.0
 3Pe
>>1
 1De
>>1
 1Se
>>1

 There are   6 orbitals as follows:
   1s  2s  2p  3s  3p  3d
 Enter orbitals to be varied: (ALL,NONE,SOME,NIT=,comma delimited list)
>>all
 Enter those that are spectroscopic
>>1s,2s,2p

 Default electron parameters ? (Y/N) 
>>y
        WAVE FUNCTIONS NOT FOUND FOR  3s
        WAVE FUNCTIONS NOT FOUND FOR  3p
        WAVE FUNCTIONS NOT FOUND FOR  3d

Default values for other parameters? (Y/N) 
>>n

 Default values (NO,REL,STRONG) ? (Y/N) 
>>y

 O         8.   220  6  6  0  F

 Default values for PRINT, CFGTOL, SCFTOL ? (Y/N) 
>>n
 Input free FORMAT(L, F, F) 
>>.f.,1.e-50,1.e-50

 Default values for NSCF, IC ? (Y/N) 
>>n
 Input free FORMAT(I, I) 
>>500,0

Default values for ACFG,LD,TRACE? (Y/N) 
>>y

............................

          ITERATION NUMBER 500
          ----------------

          CONVERGENCE CRITERIA:ENERGY  (CFGTOL)            =  1.0D-50
                              :FUNCTION(SCFTOL*SQRT(Z*NWF))=  6.9D-50

          E( 2s 1s) =     0.00788   E( 1s 2s) =     0.00759
          E( 3s 1s) =   -64.35727   E( 1s 3s) =    -0.01045
          E( 3s 2s) =    39.20184   E( 2s 3s) =     0.00661
          E( 3p 2p) =    36.90831   E( 2p 3p) =     0.00972

                     EL         ED             AZ           NORM       DPM
      < 1s| 2s>= 1.1D-07
      < 1s| 3s>= 8.7D-07
                     1s     44.6086057     43.2685679   0.9999996    1.90D-06
      < 2s| 3s>= 1.7D-06
      < 1s| 2s>= 3.5D-07
                     2s      5.1075445     10.8726274   0.9999979    4.17D-06
      < 2p| 3p>=-1.5D-06
                     2p      4.0016767     17.7222580   1.0000006    3.26D-06
      < 1s| 3s>= 2.4D-07
      < 2s| 3s>= 9.2D-07
                     3s    125.0208057     66.4580843   1.0000000**  9.31D-07
      < 2p| 3p>=-1.1D-06
                     3p    123.0829553    292.4085198   1.0000000**  1.09D-06
                     3d      7.7203415     17.4541134   1.0000000**  2.10D-08

      LEAST SELF-CONSISTENT FUNCTION IS  2s :WEIGHTED MAXIMUM CHANGE =  5.79D-06

 SCF ITERATIONS HAVE CONVERGED TO THE ABOVE ACCURACY

         ETOTAL=   -73.21750631   Loops,DeltaE,Res.:   5  9.714D-17  1.071D-08
      1  0.9849375      2  0.0007091      3 -0.0014478      4  0.0251114
      5  0.0001940      6 -0.0013826      7  0.0783650      8 -0.0474389

         ETOTAL=   -73.11718659   Loops,DeltaE,Res.:   5  3.886D-16  1.330D-08
      1  0.9847857      2  0.0009357      3  0.0006838      4 -0.0018516
      5  0.0299345      6 -0.0010101      7  0.0052231      8 -0.0109821

         ETOTAL=   -73.00594496   Loops,DeltaE,Res.:   5  2.276D-15  6.357D-09
      1  0.9651954      2  0.0013254      3  0.0012129      4 -0.0023612
      5  0.0597631      6 -0.0007850      7 -0.0195982      8  0.0017027

         Sum of ETOTAL :     -73.11354595
  DeltaE =  -7.6754247402277542E-010 Sum_Energy =   -73.113545951588947     


     ENERGY (a.u.)
     ------
           Total                 -73.113545952
           Potential            -146.226924757
           Kinetic                73.113378805
           Ratio                   2.000002286

*******************************************************************************
*         SAVE OUTPUT FILES                                                   * 
*******************************************************************************

>>cp wfn.out even3.w
>>cp cfg.inp even3.c
>>cp 3Pe.l even3Pe.l
>>cp 1De.l even1De.l
>>cp 1Se.l even1Se.l
>>cp summry even3.s


*******************************************************************************
*         RUN LSGEN TO GENERATE N = 3 SD-MR BREIT CONFIGURATION LIST          *
*         OUTPUT FILES: clist.out, clist.log                                  *
*******************************************************************************

>>lsgen

 New list, add to existing list, expand existing list, optimized sorting, restored order or quit? (*/a/e/s/r/q)

 Breit or MCHF? (B/*)
>>B
 Default, symmetry or user specified ordering? (*/s/u)
>>
 Highest principal quantum number, n? (1..15)
>>3
 Highest orbital angular momentum, l? (s..d)
>>d
 Are all these nl-subshells active? (n/*)
>>
 Limitations on population of n-subshells? (y/*)
>>
 Highest n-number in reference configuration? (1..3)
>>2
 Number of electrons in 1s? (0..2)
>>2
 Number of electrons in 2s? (0..2)
>>2
 Number of electrons in 2p? (0..6)
>>2
 Maximum 2*J-value? (0..)
>>4
 Minimum 2*J-value? (0..4)
>>0
 Maximum (2*S+1)-value? (1..9)
>>9
 Minimum (2*S+1)-value? (1..9)
>>1
 Maximum resulting angular momentum? (S..N/N=*)
>>
 Minimum resulting angular momentum? (S..N/S=*)
>>
 Number of excitations = ? (0..6)
>>2
 709 configuration states have been generated.
 Generate a second list? (y/*)
>>y
 Highest n-number? (1..15)
>>3
 Highest l-number? (s..d)
>>d
 Are all these nl-subshells active? (n/*)
>>
 Limitations on population of n-subshells? (y/*)
>>
 Highest n-number in reference configuration? (1..3)
>>2
 Number of electrons in 1s? (0..2)
>>2
 Number of electrons in 2s? (0..2)
>>0
 Number of electrons in 2p? (0..6)
>>4
 Maximum 2*J-value? (0..)
>>4
 Minimum 2*J-value? (0..4)
>>0
 Maximum (2*S+1)-value? (1..9)
>>9
 Minimum (2*S+1)-value? (1..9)
>>1
 Maximum resulting angular momentum? (S..N/N=*)
>>
 Minimum resulting angular momentum? (S..N/S=*)
>>
 Number of excitations = ? (0..6)
>>2
 491 configuration states have been generated.
 1019 configuration states in the final list.
 The merged file is called clist.out.

*******************************************************************************
*         COPY FILES                                                          *
*         IT IS ADVISABLE TO SAVE THE LSGEN LOG-FILE TO HAVE A RECORD ON      *
*         HOW THE CONFIGURATION LISTS GENERATION WAS DONE                     *
*******************************************************************************

>>cp clist.log even3BP.log
>>cp clist.out even3BP.c

*******************************************************************************
*         COPY WAVE FUNCTION FILES TO PREPARE FOR THE BREIT-PAULI RUN         *
*******************************************************************************

>>cp even3.w even3BP.w

*******************************************************************************
*         RUN CI BREIT-PAULI                                                  *
*         INPUT FILES: even3BP.c, even3BP.w                                   *
*         OUTPUT FILES: even3BP.j                                             *
*******************************************************************************

>>bpci

 Enter ATOM, relativistic (Y/N) with mass correction (Y/N)
>>even3BP,y,n
  Restarting (Y/y) ?
>>n
  Use existing Matrix and <atom>.l/j initial guess (Y/y)?
>>n

  Enter Maximum and minimum values of 2*J
>>4,0

 Enter eigenvalues: one line per term, eigenvalues separated by commas
2*J =  4
>>1,2
2*J =  2
>>1
2*J =  0
>>1,2

                    =======================
                     B R E I T - P A U L I 
                    =======================

 Indicate the type of calculation 
 0 => non-relativistic Hamiltonian only;
 1 => one or more relativistic operators only;
 2 => non-relativistic operators and selected relativistic:  
>>2
 All relativistic operators ? (Y/N) 
>>y

 THE TYPE OF CALCULATION IS DEFINED BY THE FOLLOWING PARAMETERS - 
      BREIT-PAULI OPERATORS             IREL   = 2
      PHASE CONVENTION PARAMETER        ICSTAS = 1

1

                             ---------------------
                             THE CONFIGURATION SET
                             ---------------------

 STATE  (WITH    1019 CONGIGURATIONS):
 ------------------------------------


 THERE ARE  6 ORBITALS AS FOLLOWS:

       1s  2s  2p  3s  3p  3d

 THERE ARE  0 CLOSED SUBSHELLS COMMON TO ALL CONFIGURATIONS AS FOLLOWS:


 All Interactions? (Y/N): 
>>y

 Default Rydberg constant (y/n)
>>y

..................

 Summary of Davidson Performance
 ===============================
 Number of Iterations:            9
 Shifted Eigval's: -0.37599256790146363      -0.16327545617018602     
 Delta Lambda:       3.4416913763379853E-015   1.9706458687096529E-015
 Residuals:          3.4461265110098795E-009   6.1164724733937687E-010


     2 Eigenvalues found
 Finished with Davidson
 Leaving LSJMAT
 onlydvd F
 ILS =           0
 Finished with the file









*******************************************************************************
*          RUN HF FOR AVERAGE of 1s(2)2s2p(3)                                 *
*          OUTPUT FILE: wfn.out, hf.log                                       *
*******************************************************************************

>>HF

                      =============================
                       H A R T R E E - F O C K . 86
                      =============================


               THE DIMENSIONS FOR THE CURRENT VERSION ARE:
                          NWF= 20        NO=220



 START OF CASE
 =============


 Enter ATOM,TERM,Z
 Examples: O,3P,8. or Oxygen,AV,8.
>>O,AV,8.

 List the CLOSED shells in the fields indicated (blank line if none)
 ... ... ... ... ... ... ... ... etc.
>>1s 

 Enter electrons outside CLOSED shells (blank line if none)
 Example: 2s(1)2p(3)
>>2s(1)2p(3)

 There are   3 orbitals as follows:
   1s  2s  2p

 Orbitals to be varied: ALL/NONE/=i (last i)/comma delimited list/H
>>all

 Default electron parameters ? (Y/N/H) 
>>y
          1s     1.00     44.538    43.351    SCALED O     AV_E      
          2s     3.00      5.081    10.173    SCALED O     AV_E      
          2p     4.50      3.925    17.791    SCALED O     AV_E      

 Default values for remaining parameters? (Y/N/H) 
>>y

.....................

          ITERATION NUMBER  6
          ----------------

          SCF CONVERGENCE CRITERIA (SCFTOL*SQRT(Z*NWF)) =   1.6D-06

          C( 1s 2s) =     0.00000   V( 1s 2s) =   -19.34502   EPS =-0.000000
          E( 2s 1s) =     0.05660   E( 1s 2s) =     0.02830

                     EL         ED             AZ           NORM       DPM
                     1s     44.4526692     43.1553314   1.0000000    1.11D-09
                     2s      5.1440618     10.8980587   1.0000000    7.87D-09
                     2p      3.8388851     17.7827023   1.0000000    3.11D-09


      < 1s| 2s>= 5.0D-09


     TOTAL ENERGY (a.u.)
     ----- ------
           Non-Relativistic      -72.47494137    Kinetic       72.47494129
           Relativistic Shift     -0.04744299    Potential   -144.94988266
           Relativistic          -72.52238435    Ratio        -2.000000001

 Additional parameters ? (Y/N/H) 
n

 Do you wish to continue along the sequence ? 
n


 END OF CASE
 ===========
 
*******************************************************************************
*         COPY FILES                                                          *
*         Keep a copy of the wfn.out in oddDF.w file for reference            *
*         Keep a copy of hf.log in oddDF.log                                  *
*******************************************************************************

cp wfn.out wfn.inp
cp wfn.out oddDF.w
cp hf.log oddDF.log

*******************************************************************************
*         RUN CSFEXCITATIONS TO GENERATE SD-MR EXPANSIONS FOR EACH LS TERM    *
*         AT THE END CONCATENATE FILES TO cfg.inp                             *
*******************************************************************************

*******************************************************************************
*         RUN FOR 3D                                                          *
*******************************************************************************

>>csfexcitation

 CSFEXCITATION
 This program creates excitation input to LSGEN
 Configurations should be entered in spectroscopic notation
 with occupation numbers and indications if orbitals are
 closed (c), inactive (i), active (*) or has a minimal
 occupation e.g. 1s(2,1)2s(2,*)
 Outputfiles: excitationdata.sh, csfexcitation.log

 Enter list of (maximum 100) configurations. End list with a blank line or an astersik (*).

 Give configuration           1
>>1s(2,*)2s(1,*)2p(3,*)
 Give configuration           2
>>
 Give set of active orbitals in a comma delimited list ordered by l-symmetry, e.g. 5s,4p,3d
>>3s,3p,3d
 Resulting term? (1S, 3P, etc.)
>>3D
 Number of excitations (if negative number e.g. -2, correlation 
 orbitals will always be doubly occupied)                        
>>2
 Generate more lists ? (y/n)
>>n

*******************************************************************************
*         RUN THE SHELL SCRIPT excitationdata.sh PRODUCED BY CSFEXCITATION    *
*******************************************************************************

>>./excitationdata.sh

..........

 160 configuration states in the final list.
 The merged file is called clist.out.

*******************************************************************************
*         SAVE IN odd33D                                                     *
*******************************************************************************

>>mv clist.out odd33D

*******************************************************************************
*         RUN FOR 3P                                                          *
*******************************************************************************

>>csfexcitation

 CSFEXCITATION
 This program creates excitation input to LSGEN
 Configurations should be entered in spectroscopic notation
 with occupation numbers and indications if orbitals are
 closed (c), inactive (i), active (*) or has a minimal
 occupation e.g. 1s(2,1)2s(2,*)
 Outputfiles: excitationdata.sh, csfexcitation.log

 Enter list of (maximum 100) configurations. End list with a blank line or an astersik (*).

 Give configuration           1
>>1s(2,*)2s(1,*)2p(3,*)
 Give configuration           2
>>
 Give set of active orbitals in a comma delimited list ordered by l-symmetry, e.g. 5s,4p,3d
>>3s,3p,3d
 Resulting term? (1S, 3P, etc.)
>>3P
 Number of excitations (if negative number e.g. -2, correlation 
 orbitals will always be doubly occupied)                        
>>2
 Generate more lists ? (y/n)
>>n

*******************************************************************************
*         RUN THE SHELL SCRIPT excitationdata.sh PRODUCED BY CSFEXCITATION    *
*******************************************************************************

>>./excitationdata.sh

..........

 151 configuration states in the final list.
 The merged file is called clist.out.

*******************************************************************************
*         SAVE IN odd33P                                                     *
*******************************************************************************

>>mv clist.out odd33P

*******************************************************************************
*         RUN FOR 1D                                                          *
*******************************************************************************

>>csfexcitation

 CSFEXCITATION
 This program creates excitation input to LSGEN
 Configurations should be entered in spectroscopic notation
 with occupation numbers and indications if orbitals are
 closed (c), inactive (i), active (*) or has a minimal
 occupation e.g. 1s(2,1)2s(2,*)
 Outputfiles: excitationdata.sh, csfexcitation.log

 Enter list of (maximum 100) configurations. End list with a blank line or an astersik (*).

 Give configuration           1
>>1s(2,*)2s(1,*)2p(3,*)
 Give configuration           2
>>
 Give set of active orbitals in a comma delimited list ordered by l-symmetry, e.g. 5s,4p,3d
>>3s,3p,3d
 Resulting term? (1S, 3P, etc.)
>>1D
 Number of excitations (if negative number e.g. -2, correlation 
 orbitals will always be doubly occupied)                        
>>2
 Generate more lists ? (y/n)
>>n

*******************************************************************************
*         RUN THE SHELL SCRIPT excitationdata.sh PRODUCED BY CSFEXCITATION    *
*******************************************************************************

>>./excitationdata.sh

..........

 100 configuration states in the final list.
 The merged file is called clist.out.

*******************************************************************************
*         SAVE IN odd31D                                                     *
*******************************************************************************

>>mv clist.out odd31D


*******************************************************************************
*         RUN FOR 3S                                                          *
*******************************************************************************

>>csfexcitation

 CSFEXCITATION
 This program creates excitation input to LSGEN
 Configurations should be entered in spectroscopic notation
 with occupation numbers and indications if orbitals are
 closed (c), inactive (i), active (*) or has a minimal
 occupation e.g. 1s(2,1)2s(2,*)
 Outputfiles: excitationdata.sh, csfexcitation.log

 Enter list of (maximum 100) configurations. End list with a blank line or an astersik (*).

 Give configuration           1
>>1s(2,*)2s(1,*)2p(3,*)
 Give configuration           2
>>
 Give set of active orbitals in a comma delimited list ordered by l-symmetry, e.g. 5s,4p,3d
>>3s,3p,3d
 Resulting term? (1S, 3P, etc.)
>>3S
 Number of excitations (if negative number e.g. -2, correlation 
 orbitals will always be doubly occupied)                        
>>2
 Generate more lists ? (y/n)
>>n

*******************************************************************************
*         RUN THE SHELL SCRIPT excitationdata.sh PRODUCED BY CSFEXCITATION    *
*******************************************************************************

>>./excitationdata.sh

..........

 68 configuration states in the final list.
 The merged file is called clist.out.

*******************************************************************************
*         SAVE IN odd33S                                                     *
*******************************************************************************

>>mv clist.out odd33S

*******************************************************************************
*         RUN FOR 1P                                                          *
*******************************************************************************

>>csfexcitation

 CSFEXCITATION
 This program creates excitation input to LSGEN
 Configurations should be entered in spectroscopic notation
 with occupation numbers and indications if orbitals are
 closed (c), inactive (i), active (*) or has a minimal
 occupation e.g. 1s(2,1)2s(2,*)
 Outputfiles: excitationdata.sh, csfexcitation.log

 Enter list of (maximum 100) configurations. End list with a blank line or an astersik (*).

 Give configuration           1
>>1s(2,*)2s(1,*)2p(3,*)
 Give configuration           2
>>
 Give set of active orbitals in a comma delimited list ordered by l-symmetry, e.g. 5s,4p,3d
>>3s,3p,3d
 Resulting term? (1S, 3P, etc.)
>>1P
 Number of excitations (if negative number e.g. -2, correlation 
 orbitals will always be doubly occupied)                        
>>2
 Generate more lists ? (y/n)
>>n

*******************************************************************************
*         RUN THE SHELL SCRIPT excitationdata.sh PRODUCED BY CSFEXCITATION    *
*******************************************************************************

>>./excitationdata.sh

..........

 105 configuration states in the final list.
 The merged file is called clist.out.

*******************************************************************************
*         SAVE IN odd31P                                                      *
*******************************************************************************

>>mv clist.out odd31P

*******************************************************************************
*         CONCATENATE THE FILES                                               *
*******************************************************************************

>>cat odd33D odd33P odd31D odd33S odd31P > cfg.inp

*******************************************************************************
*         RUN NONH TO GENERATE ENERGY EXPRESSION                              *
*         INPUT FILES: cfg.inp                                                *
*         OUTPUT FILES: cfg.h, yint.lst, c.lst, ih.=n.lst                     *
*******************************************************************************

>>nonh

                       ===============================
                             N O N H       2000
                       ===============================


 input file is cfg.inp ...


 THERE ARE  6 ORBITALS AS FOLLOWS:

       1s  2s  2p  3s  3p  3d

 THERE ARE  0 CLOSED SUBSHELLS COMMON TO ALL CONFIGURATIONS AS FOLLOWS:


 Allocating space for          214  integrals
 processing 3Do with          160 configurations
    jb =         100
    4810 non-zero matrix elements
    1639 NF    2667 NG    7287 NR     862 NL
   12455 Total number of integrals
 processing 3Po with          151 configurations
    jb =         100
    4485 non-zero matrix elements
    1502 NF    2213 NG    7035 NR     836 NL
   11586 Total number of integrals
 processing 1Do with          100 configurations
    jb =         100
    2163 non-zero matrix elements
    1011 NF    1178 NG    3593 NR     529 NL
    6311 Total number of integrals
 processing 3So with           68 configurations
    1182 non-zero matrix elements
     631 NF     697 NG    2040 NR     370 NL
    3738 Total number of integrals
 processing 1Po with          105 configurations
    jb =         100
    2346 non-zero matrix elements
    1009 NF    1137 NG    4001 NR     570 NL
    6717 Total number of integrals

 end-of-file clist!!!

*******************************************************************************
*         RUN MCHF TO OBTAIN SELF CONSISTENT SOLUTIONS                        *
*         INPUT FILES: wfn.inp (optional), cfg.inp, angular files             *
*         OUTPUT FILES: wfn.out, 3Pe.l, 1De.l 1Se.l summry                    *
*                                                                             *
*         NOTE1: Mixing coefficients written to a files named by LS-symmetry  *
*         and parity                                                          *
*                                                                             *
*         NOTE2: We force the program to iterate to higher precision than     *
*         default which is a good thing for small calculations                *
*******************************************************************************

>>mchf

                    =======================
                            M C H F  ... 2000   
                    =======================




          THE DIMENSIONS FOR THE CURRENT VERSION ARE:
             NWD= 60        NO=220    Lagrange Multipliers=800



 START OF CASE
 =============


 ATOM, Z in FORMAT(A, F) : 
>>O,8.
 cfg.inp has configurations for            3  terms

 Enter eigenvalues and weights: one line per term, eigenvalues with weights
 in parenthesis and separated by commas, default is 1.0
 3Do
>>1
 3Po
>>1
 1Do
>>1
 3So
>>1
 1Po
>>1 

 There are   6 orbitals as follows:
   1s  2s  2p  3s  3p  3d
 Enter orbitals to be varied: (ALL,NONE,SOME,NIT=,comma delimited list)
>>all
 Enter those that are spectroscopic
>>1s,2s,2p

 Default electron parameters ? (Y/N) 

>>y
        WAVE FUNCTIONS NOT FOUND FOR  3s
        WAVE FUNCTIONS NOT FOUND FOR  3p
        WAVE FUNCTIONS NOT FOUND FOR  3d

Default values for other parameters? (Y/N) 
>>n

 Default values (NO,REL,STRONG) ? (Y/N) 
>>y

 O         8.   220  6  6  0  F

 Default values for PRINT, CFGTOL, SCFTOL ? (Y/N) 
>>n
 Input free FORMAT(L, F, F) 
>>.f.,1.e-50,1.e-50

 Default values for NSCF, IC ? (Y/N) 
>>n
 Input free FORMAT(I, I) 
>>500,0

Default values for ACFG,LD,TRACE? (Y/N) 
>>y

............................

          ITERATION NUMBER 500
          ----------------

          CONVERGENCE CRITERIA:ENERGY  (CFGTOL)            =  1.0D-50
                              :FUNCTION(SCFTOL*SQRT(Z*NWF))=  6.9D-50

          E( 2s 1s) =     0.06046   E( 1s 2s) =     0.03014
          E( 3s 1s) =  -112.20232   E( 1s 3s) =    -0.01934
          E( 3s 2s) =    37.24175   E( 2s 3s) =     0.01288
          E( 3p 2p) =    10.57769   E( 2p 3p) =     0.00222

                     EL         ED             AZ           NORM       DPM
      < 1s| 2s>= 4.4D-07
      < 1s| 3s>= 2.4D-08
                     1s     44.5598375     43.2430058   1.0000002    4.46D-07
      < 2s| 3s>= 2.0D-07
      < 1s| 2s>= 1.1D-06
                     2s      4.9397943     10.8976675   0.9999980    2.45D-06
      < 2p| 3p>= 7.0D-07
                     2p      3.8279380     18.2169049   0.9999990    1.43D-06
      < 1s| 3s>= 2.1D-06
      < 2s| 3s>= 4.4D-07
                     3s    117.5699862     64.2213384   1.0000000**  4.50D-06
      < 2p| 3p>= 4.1D-07
                     3p    108.9572949    259.4329605   1.0000000**  4.21D-07
                     3d      5.9273558     16.5774696   1.0000000**  1.66D-08

      LEAST SELF-CONSISTENT FUNCTION IS  2p :WEIGHTED MAXIMUM CHANGE =  2.46D-06

 SCF ITERATIONS HAVE CONVERGED TO THE ABOVE ACCURACY

         ETOTAL=   -72.66319592   Loops,DeltaE,Res.:   5  1.776D-15  1.172D-09
      1  0.0950436      2  0.9914537      3  0.0032167      4  0.0022331
      5 -0.0013008      6 -0.0000631      7 -0.0005015      8  0.0007031

         ETOTAL=   -72.55129214   Loops,DeltaE,Res.:   5  0.000D+00  1.019D-09
      1  0.0008201      2 -0.0766761      3  0.9937388      4 -0.0022315
      5  0.0022502      6 -0.0022397      7  0.0010188      8  0.0011151

         ETOTAL=   -72.34271426   Loops,DeltaE,Res.:   5  3.109D-15  1.485D-09
      1 -0.1165446      2  0.9843832      3 -0.0004768      4 -0.0032576
      5  0.0009593      6 -0.0000297      7 -0.0029173      8  0.0023311

         ETOTAL=   -72.30362983   Loops,DeltaE,Res.:   5  5.551D-16  1.078D-09
      1  0.9888834      2 -0.0018892      3  0.0021459      4 -0.0026835
      5 -0.0019315      6  0.0361699      7  0.0219719      8 -0.0024910

         ETOTAL=   -72.22169734   Loops,DeltaE,Res.:   5  0.000D+00  1.145D-09
      1 -0.0009225      2  0.0681925      3  0.9905430      4  0.0002057
      5 -0.0007020      6  0.0030295      7  0.0011131      8 -0.0006380

         Sum of ETOTAL :     -72.41650590
  DeltaE =  -1.0173550890613114E-010 Sum_Energy =   -72.416505898330826     


     ENERGY (a.u.)
     ------
           Total                 -72.416505898
           Potential            -144.832966144
           Kinetic                72.416460246
           Ratio                   2.000000630

*******************************************************************************
*         SAVE OUTPUT FILES                                                   * 
*******************************************************************************

>>cp wfn.out odd3.w
>>cp cfg.inp odd3.c
>>cp 3Do.l odd3Do.l
>>cp 3Po.l odd3Po.l
>>cp 1Do.l odd1Do.l
>>cp 3So.l odd3So.l
>>cp 1Po.l odd1Po.l
>>cp summry odd3.s


*******************************************************************************
*         RUN LSGEN TO GENERATE N = 3 SD-MR BREIT CONFIGURATION LIST          *
*         OUTPUT FILES: clist.out, clist.log                                  *
*******************************************************************************

>>lsgen

 New list, add to existing list, expand existing list, optimized sorting, restored order or quit? (*/a/e/s/r/q)

 Breit or MCHF? (B/*)
>>B
 Default, symmetry or user specified ordering? (*/s/u)
>>
 Highest principal quantum number, n? (1..15)
>>3
 Highest orbital angular momentum, l? (s..d)
>>d
 Are all these nl-subshells active? (n/*)
>>
 Limitations on population of n-subshells? (y/*)
>>
 Highest n-number in reference configuration? (1..3)
>>2
 Number of electrons in 1s? (0..2)
>>2
 Number of electrons in 2s? (0..2)
>>1
 Number of electrons in 2p? (0..6)
>>3
 Maximum 2*J-value? (0..)
>>6
 Minimum 2*J-value? (0..4)
>>0
 Maximum (2*S+1)-value? (1..9)
>>9
 Minimum (2*S+1)-value? (1..9)
>>1
 Maximum resulting angular momentum? (S..N/N=*)
>>
 Minimum resulting angular momentum? (S..N/S=*)
>>
 Number of excitations = ? (0..6)
>>2
 1075 configuration states have been generated.
 Generate a second list? (y/*)
>>n
 1075 configuration states in the final list.
 The generated file is called clist.out.


*******************************************************************************
*         COPY FILES                                                          *
*         IT IS ADVISABLE TO SAVE THE LSGEN LOG-FILE TO HAVE A RECORD ON      *
*         HOW THE CONFIGURATION LISTS GENERATION WAS DONE                     *
*******************************************************************************

>>cp clist.log odd3BP.log
>>cp clist.out odd3BP.c

*******************************************************************************
*         COPY WAVE FUNCTION FILES TO PREPARE FOR THE BREIT-PAULI RUN         *
*******************************************************************************

>>cp odd3.w odd3BP.w

*******************************************************************************
*         RUN CI BREIT-PAULI                                                  *
*         INPUT FILES: odd3BP.c, odd3BP.w                                     *
*         OUTPUT FILES: odd3BP.j                                              *
*******************************************************************************

>>bpci

 Enter ATOM, relativistic (Y/N) with mass correction (Y/N)
>>odd3BP,y,n
  Restarting (Y/y) ?
>>n
  Use existing Matrix and <atom>.l/j initial guess (Y/y)?
>>n

  Enter Maximum and minimum values of 2*J
>>4,0

 Enter eigenvalues: one line per term, eigenvalues separated by commas
2*J =  6
>>1
2*J =  4
>>1,2,3,4
2*J =  2
>>1,2,3,4
2*J =  0
>>1

                    =======================
                     B R E I T - P A U L I 
                    =======================

 Indicate the type of calculation 
 0 => non-relativistic Hamiltonian only;
 1 => one or more relativistic operators only;
 2 => non-relativistic operators and selected relativistic:  
>>2
 All relativistic operators ? (Y/N) 
>>y

 THE TYPE OF CALCULATION IS DEFINED BY THE FOLLOWING PARAMETERS - 
      BREIT-PAULI OPERATORS             IREL   = 2
      PHASE CONVENTION PARAMETER        ICSTAS = 1

1

                             ---------------------
                             THE CONFIGURATION SET
                             ---------------------

 STATE  (WITH    1075 CONGIGURATIONS):
 ------------------------------------


 THERE ARE  6 ORBITALS AS FOLLOWS:

       1s  2s  2p  3s  3p  3d

 THERE ARE  0 CLOSED SUBSHELLS COMMON TO ALL CONFIGURATIONS AS FOLLOWS:


 All Interactions? (Y/N): 
>>y

 Default Rydberg constant (y/n)
>>y

..................
 
  Summary of Davidson Performance
 ===============================
 Number of Iterations:            8
 Shifted Eigval's:  -25.848333706504089     
 Delta Lambda:       0.0000000000000000     
 Residuals:          4.6248713217869881E-009


     1 Eigenvalues found
 Finished with Davidson
 Leaving LSJMAT
 onlydvd F
 ILS =           0
 Finished with the file
 
*******************************************************************************
*         RUN BIOTR TO COMPUTE TRANSITION RATES                               *
*         INPUT FILES: even3BP.c, even3BP.w, even3BP.j                        *
*                      odd3BP.c, odd3BP.w, odd3BP.j                           *
*         OUTPUT FILES: even3BP.odd3BP.lsj                                    *
******************************************************************************* 
 
 
>>biotr

                    ========================
                      T R A N S B I O  99 
                    ========================


  Name of Initial State
>>even3BP
  Name of Final State
>>odd3BP
  intermediate printing (y or n) ?  
>>n
  Relativistic calculation ? (y/n) 
>>y
  Type of transition ? (E1, E2, M1, M2, .. or *) 
>>E1
 
 .............................
 
------------------------------------------------------
Pair number  27



 Initial CSF : 1s(2).2s(2).2p(2)1S0_1S                            J = 0.0
 Final   CSF : 1s(2).2s_2S.2p(3)4S3_3S                            J = 1.0

 2*j =     0 lbl =     1 total energy =      -73.0557055
 2*j =     2 lbl =    13 total energy =      -72.3510118


          LENGTH   FORMALISM: 
          -------- ----------


          SL                                       =   1.5711546D-05
          FINAL OSCILLATOR STRENGTH (GF)           =   7.3809621D-06
          TRANSITION PROBABILITY IN EMISSION (Aki) =   3.9253079D+04




          VELOCITY FORMALISM: 
          -------- ----------


          SV                                       =   2.0465417D-05
          FINAL OSCILLATOR STRENGTH (GF)           =   9.6142333D-06
          TRANSITION PROBABILITY IN EMISSION (Aki) =   5.1129955D+04


  Type of transition ? (E1, E2, M1, M2, .. or *) 
>>*
STOP  END OF CASE

 

\end{verbatim}                            
\chapter{Structure of the output files}
In this chapter we have a look at the output files so that we can interpret the most important data.
\section{HF}
Below is the hf.log file from the $1s^22s2p^3$ run.
\begin{verbatim}
1                                                                                                                                                             
         HARTREE-FOCK WAVE FUNCTIONS FOR  O     AV     Z =  8.0 

              Core =  1s(   2)  
     Configuration =  2s(   1)  2p(   3)    

         INPUT DATA
         ----- ----

             WAVE FUNCTION  PROCEDURE
                 NL  SIGMA METH ACC OPT 


       1   1s  1  0    1.0   1 0.0  -1  
       2   2s  2  0    2.0   1 0.0  -1  
       3   2p  2  1    4.0   1 0.0  -1  

        INITIAL ESTIMATES 

          NL    SIGMA      E(NL)    AZ(NL)    FUNCTIONS


          1s     1.00     44.545    43.195    SCALED O     AV    
          2s     2.00      4.969    10.763    SCALED O     AV    
          2p     4.00      3.989    17.947    SCALED O     AV    

          NUMBER OF FUNCTIONS ITERATED          =     3   
          MAXIMUM WEIGHTED CHANGE IN FUNCTIONS  =  0.79D-08

                        ATOM O        TERM AV    

  nl       E(nl)        I(nl)       KE(nl)        Rel(nl)   S(nl)       Az(nl)
  1s    44.4526692   -31.923618    29.208718    -0.022348   0.449     43.155331
  2s     5.1440618    -7.356525     3.842790    -0.003202   2.163     10.898059
  2p     3.8388851    -6.987698     3.404906    -0.000939   3.024     17.782702


  nl      Delta(R)     1/R**3       1/R         R        R**2
  1s       148.204       0.0000    7.64154   0.19864    0.05319
  2s         9.451       0.0000    1.39991   1.02798    1.25223
  2p         0.000       6.9651    1.29908   1.00479    1.25465


     TOTAL ENERGY (a.u.)
     ----- ------
           Non-Relativistic      -72.47494137    Kinetic       72.47494129
           Relativistic Shift     -0.04744299    Potential   -144.94988266
           Relativistic          -72.52238435    Ratio        -2.000000001         
\end{verbatim}
The most interesting quantities are \verb+E(nl)+, the orbital energy, \verb+S(nl)+ the screening, \verb+Az(nl)+ the slope at the origin. \verb+R+ is the one-electron expectation value of $r$. It is the measure of the mean radius of the $P(r)$ orbital. Remaining quantities are defined and discussed in the MCHF book. At the end total energies are displayed along with the value of the relativistic shift energy correction. 
\section{The cfg.inp files}
Below is a part of the \verb+odd33D+ file.
\begin{verbatim}
  1s( 2)  2s( 2)  2p( 1)  3d( 1)
 1S0 1S0 2P1 2D1 1S  2P  3D  
  1s( 2)  2s( 1)  2p( 3)
 1S0 2S1 2D3 2S  3D 
  1s( 2)  2s( 1)  2p( 2)  3p( 1)
 1S0 2S1 1D2 2P1 2S  2D  3D  
  1s( 2)  2s( 1)  2p( 2)  3p( 1)
 1S0 2S1 3P2 2P1 2S  2P  3D  
  1s( 2)  2s( 1)  2p( 2)  3p( 1)
 1S0 2S1 3P2 2P1 2S  4P  3D  
  1s( 2)  2s( 1)  2p( 1)  3s( 1)  3d( 1)
 1S0 2S1 2P1 2S1 2D1 2S  1P  2P  3D  
  1s( 2)  2s( 1)  2p( 1)  3s( 1)  3d( 1)
 1S0 2S1 2P1 2S1 2D1 2S  3P  2P  3D  
  1s( 2)  2s( 1)  2p( 1)  3s( 1)  3d( 1)
 1S0 2S1 2P1 2S1 2D1 2S  3P  4P  3D  

             .....................
             
  2s( 1)  2p( 3)  3d( 2)
 2S1 2D3 3F2 3D  3D
  2s( 1)  2p( 3)  3d( 2)
 2S1 4S3 1D2 3S  3D
*            
\end{verbatim}
Each CSF is defined on two lines. The first line gives the configuration and the second line gives the coupling tree. The coupling tree for the first CSF is: \verb+ 1S0 1S0 2P1 2D1 1S  2P  3D+.
From left to right: \verb+1S0+ is the coupling including seniority of the $1s^2$ subshell, \verb+1S0+ is the coupling including seniority of the $2s^2$ subshell, \verb+2P1+ is the coupling including seniority of the $2p$ subshell,
\verb+2D1+ is the coupling including seniority of the $3d$ subshell.  \verb+1S  2P  3D+ then shows how the quantum numbers of the subshells are coupled together from left to right to give the final \verb+3D+
\section{The output files from MCHF}
Below is the file \verb+summry+ from the run for $2s2p^3$. 
\begin{verbatim}
1                                                                                                                                                                                             


         HARTREE-FOCK WAVE FUNCTIONS FOR  O     AV_E   Z =  8.0

              CORE = 


           CONFIGURATION                                     WEIGHT




         INPUT DATA
         ----- ----

             WAVE FUNCTION  PROCEDURE
                 NL  SIGMA METH ACC OPT



       1   1s  1  0    0.0   1 0.0  -1
       2   2s  2  0    0.0   1 0.0  -1
       3   2p  2  1    0.0   1 0.0  -1
       4   3s  3  0    0.0   3 0.0  -1
       5   3p  3  1    0.0   3 0.0  -1
       6   3d  3  2    0.0   3 0.0  -1



        INITIAL ESTIMATES 

          NL    SIGMA      E(NL)    AZ(NL)    FUNCTIONS


          1s     0.00     44.453    43.155    SCALED O     AV        
          2s     0.00      5.144    10.898    SCALED O     AV        
          2p     0.00      3.839    17.783    SCALED O     AV        
          3s     0.00      0.000     8.709    SCREENED HYDROGENIC    
          3p     0.00      0.000    21.897    SCREENED HYDROGENIC    
          3d     0.00      0.000    13.057    SCREENED HYDROGENIC    

 SCF ITERATIONS HAVE CONVERGED TO THE ABOVE ACCURACY


               Some WaveFunction Properties

   Term 3Do
 Mean radius             =     4.45705988
 Mean square radius      =     5.16793250
 Mean R.R parameter      =     3.32383707
 Isotope Shift parameter =    -4.17289084

                                                                                                                                                                            1,1           Top
               Some WaveFunction Properties

   Term 3Po
 Mean radius             =     4.45642914
 Mean square radius      =     5.16669598
 Mean R.R parameter      =     3.41274468
 Isotope Shift parameter =    -4.04764169


               Some WaveFunction Properties

   Term 1Do
 Mean radius             =     4.45899901
 Mean square radius      =     5.17152972
 Mean R.R parameter      =     4.36607816
 Isotope Shift parameter =    -3.91263329


               Some WaveFunction Properties

   Term 3So
 Mean radius             =     4.45681867
 Mean square radius      =     5.16770642
 Mean R.R parameter      =     5.09258590
 Isotope Shift parameter =    -3.55774766


               Some WaveFunction Properties

   Term 1Po
 Mean radius             =     4.45701026
 Mean square radius      =     5.16773873
 Mean R.R parameter      =     4.55080085
 Isotope Shift parameter =    -3.67939887



                        ATOM O        TERM AV_E  

                                                               MEAN VALUE OF                     ONE ELECTRON INTEGRALS
     NL       E(NL)         I(nl)     KE(nl)     RelS     S(nl)     Az(nl)
  1s    44.5598375   -31.927369    29.299493    -0.022582   0.44     43.24301
  2s     4.9397943    -7.331467     3.801104    -0.003224   2.21     10.89767
  2p     3.8279380    -6.989867     3.418299    -0.000964   3.03     18.21690
  3s   117.5699862    14.243149    78.102787    -0.267367 -28.54     64.22134
  3p   108.9572949    12.844521    51.782852    -0.139026 -30.68    259.43296
  3d     5.9273558    -3.509729     4.393348    -0.000908  -0.88     16.57747                                                                                                                 


  nl      Delta(R)     1/R**3       1/R         R        R**2
  1s       148.806       0.0000    7.65336   0.19838    0.05307 1.9990769453
  2s         9.451       0.0000    1.39157   1.03693    1.27669 0.9965600868
  2p         0.000       7.0801    1.30102   1.00533    1.25743 2.9806733240
  3s       328.208       0.0000    7.98245   0.36951    0.30093 0.0003445265
  3p         0.000     297.3674    4.86729   0.32319    0.21845 0.0006247942
  3d         0.000       1.7084    0.98788   1.18225    1.60443 0.0227203232


     ENERGY (a.u.)
     ------
           Total                 -72.416505898
           Potential            -144.832966144
           Kinetic                72.416460246
           Ratio                   2.000000630                        
\end{verbatim}
For each term the mean radius is computed as the expectation values of $\sum_i r_i$, the mean square radius is the expectation value of $\sum_i r^ 2_i$, the mean RR
parameter is the expectation value of $( \sum_i r_i)^2$. Finally, the isotope shift parameter is obtained as the
expectation value of $ - \sum_{i< j} \nabla_i \cdot \nabla_j$. As for the HF program different expectation values are given for the radial orbitals. At the en the total weighted energy of all LS terms are given.\medskip\\
Below is the file \verb+odd33Do.l+
\begin{verbatim}
  O       Z =   8.0  NEL =   0   NCFG =    160                                                                                                                                                


  2*J =    0  NUMBER =   1   
     Ssms =     -4.172890838
     2   -72.663195923  1s(2).2s_2S.2p(3)2D3_3D
 0.09504360 0.99145367 0.00321666 0.00223312-0.00130085-0.00006311-0.00050151
 0.00070311-0.00294346-0.00131389 0.00188791 0.03690362 0.01879516-0.02664271
-0.00567333 0.00122499 0.00091130 0.00609791 0.03854841-0.05864913 0.00239437
-0.00074051-0.00234218 0.00007151-0.00033710 0.00112085-0.00128884 0.00012146
 0.00023730 0.00011232 0.00103733-0.00162359 0.00136717-0.00122566-0.00173749
 0.00065755 0.00335629 0.00076740-0.00179260-0.00221321-0.00018846-0.00037952
-0.00096277 0.00129382 0.00342189-0.00195677-0.00192879 0.00106157-0.00210511
-0.00236640-0.00277370-0.00180175 0.00152471 0.00119587-0.00076677-0.00004599
 0.00009377 0.00351074-0.00005340-0.00013785-0.00000099 0.00054061-0.00108130
 0.00148476 0.00083633-0.00216838 0.00269248-0.00006759 0.00017722-0.00019519
-0.00025881 0.00026153-0.00045868-0.00064684 0.00139985-0.00120741 0.00032801
-0.00022181 0.00042567 0.00052078-0.00040657 0.00075440-0.00066470 0.00105170
-0.00090607-0.00120322 0.00203661-0.00176800-0.00049515-0.00173083-0.00173754
 0.00345180 0.00001463 0.00000306 0.00001458 0.00001316 0.00004469 0.00003841
 0.00000833-0.00011605 0.00000207 0.00000150-0.00000131-0.00000238-0.00232146
-0.00001966 0.00000323 0.00000450-0.00003012 0.00008190 0.00003713-0.00009916
 0.00002037 0.00002707-0.00090401-0.00011849-0.00022889 0.00006318 0.00002672
-0.00013692 0.00019347-0.00006561-0.00425206 0.00013408-0.00006138 0.00009443
 0.00318261 0.00105697-0.00301032-0.01092566 0.00001040 0.00003816-0.00002412
 0.00017767 0.00013388-0.00004706-0.00002762-0.00023685-0.00021230-0.00002119
-0.00000365 0.00000605 0.01450724-0.00000976 0.00000525 0.00000767-0.00001439
 0.00005652 0.00002862-0.00004147-0.00000983 0.00000485-0.00018650-0.00000418
-0.00001304 0.00000193-0.00000097-0.00000785 0.00001050-0.00000112
\end{verbatim}
The file gives the expansion coefficients of each of the CSFs that describe the wave function of the $3Do$ symmetry.
The file also gives the total energy. \verb+Ssms+ is the expectation value of $ - \sum_{i< j} \nabla_i \cdot \nabla_j$            
\section{The output file from bpci}
Below is part of the \verb+odd3BP.j+ file
\begin{verbatim}
  O       Z =   8.0  NEL =   6   NCFG =   1075                                                                                                                                                


  2*J =    6  NUMBER =   1
   Ssms=   0.0000000000    g_J=   1.3341064345  g_JLS=   1.3341064348
    12   -72.710926268  1s(2).2s_2S.2p(3)2D3_3D
 0.00000000 0.00000000 0.00000000 0.00000000-0.00003906 0.00000000 0.09511942
-0.00001682 0.00000000 0.00000000 0.00000000 0.99144638 0.00000000 0.00000000
 0.00000000 0.00000000 0.00000000 0.00000000 0.00010474 0.00000000 0.00302117
-0.00012257 0.00000000 0.00000000 0.00000000 0.00000000 0.00000000 0.00219937
 0.00000000 0.00000000-0.00100627 0.00000000-0.00011113-0.00015128 0.00000000
 0.00000000 0.00000000 0.00000000-0.00000005 0.00000000-0.00012657 0.00000040
 0.00000000 0.00000000 0.00000053 0.00000000-0.00061341-0.00000043 0.00000000
 0.00070451-0.00000003 0.00000040 0.00000034-0.00000020 0.00000000 0.00000000
 0.00000000 0.00000000 0.00000197 0.00000000-0.00296370-0.00000194 0.00000000
 0.00000000-0.00132116 0.00000000 0.00000000 0.00000000 0.00000000 0.00000000
 0.00190059 0.00000000 0.00000078-0.00000395 0.00000000 0.00000000 0.00000000
 0.00000000 0.00000064 0.00000000 0.03691234-0.00000177 0.00000267 0.00000054
 0.00000028 0.00000000 0.00000000 0.01880101 0.00000000 0.00000000 0.00000000
 0.00000000 0.00000000-0.02665431 0.00000000 0.00000527 0.00000160-0.00567808
-0.00000110-0.00000082 0.00000000 0.00000541 0.00122482-0.00000471-0.00000058
 0.00000171 0.00000502 0.00000116 0.00000000 0.00000000 0.00000000 0.00000781
 0.00000000 0.00000000 0.00000000 0.00000000-0.00002487 0.00000000 0.00607916
 0.00000260 0.00000000 0.00000000 0.00000000-0.00000731 0.00000000 0.00000000
 0.03853602 0.00001242 0.00000316-0.05861989 0.00000228 0.00000000 0.00000000
 0.00000000 0.00000000-0.00000185 0.00000000 0.00241928 0.00000308 0.00000000
 0.00000000 0.00000000 0.00000000 0.00000000-0.00074721 0.00000000 0.00000000
-0.00236775 0.00000000 0.00000168 0.00000238 0.00000000 0.00000000-0.00000049
 0.00000000 0.00007258-0.00000056 0.00000000 0.00000000-0.00000044 0.00000000
\end{verbatim}
The file gives, for each state $J$, the total energy, the quantum label, the Land\'e $g_J$ factor. This information is followed by expansion coefficients relative to the CSF basis given in \verb+odd3BP.c+. 
\section{The output file from biotra}
Below is part of the file \verb+even3BP.odd3BP.lsj+
\begin{verbatim}
  Transition between files:                                                                                                                                                                   
  even3BP    
  odd3BP    


   4  -73.26708654  1s(2).2s(2).2p(2)3P2_3P    
   6  -72.71092627  1s(2).2s_2S.2p(3)2D3_3D    
  122058.88 CM-1       819.28 ANGS(VAC)       819.28 ANGS(AIR)
 E1  S =  1.15912D+00   GF =  4.29758D-01   AKI =  6.10108D+08
          1.32563D+00         4.91491D-01          6.97749D+08


   4  -73.26708654  1s(2).2s(2).2p(2)3P2_3P    
   4  -73.00138768  1s(2).2s_2S.2p(3)4S3_5S    
   58312.16 CM-1      1714.91 ANGS(VAC)      1714.91 ANGS(AIR)
 E1  S =  5.54317D-06   GF =  9.81842D-07   AKI =  4.45381D+02
          2.36087D-05         4.18173D-06          1.89691D+03


   4  -73.26708654  1s(2).2s(2).2p(2)3P2_3P    
   4  -72.71077079  1s(2).2s_2S.2p(3)2D3_3D    
  122093.01 CM-1       819.05 ANGS(VAC)       819.05 ANGS(AIR)
 E1  S =  2.00093D-01   GF =  7.42075D-02   AKI =  1.47571D+08
          2.28524D-01         8.47514D-02          1.68539D+08


   4  -73.26708654  1s(2).2s(2).2p(2)3P2_3P    
   4  -72.59887708  1s(2).2s_2S.2p(3)2P1_3P    
  146649.99 CM-1       681.90 ANGS(VAC)       681.90 ANGS(AIR)
 E1  S =  1.04924D+00   GF =  4.67390D-01   AKI =  1.34096D+09
          9.68732D-01         4.31529D-01          1.23807D+09


   4  -73.26708654  1s(2).2s(2).2p(2)3P2_3P    
   2  -72.71072621  1s(2).2s_2S.2p(3)2D3_3D    
  122102.79 CM-1       818.98 ANGS(VAC)       818.98 ANGS(AIR)
 E1  S =  1.30594D-02   GF =  4.84367D-03   AKI =  1.60564D+07
          1.49051D-02         5.52820D-03          1.83255D+07
\end{verbatim}
The transitions are defined on the two first lines. 
The first line gives the lower state where the first quantity is $2J$ of the state, then is the total energy of the state and finally the quantum designation. The second line gives the same information for the upper state. On the third line the transition energy and the wave lengths are given. Finally, on lines four and five the line strengt, the weighted oscillator strength and the transition rate are given in length and velocity forms, respectively. 
\section{Plot radial orbitals}
The package contains the program \verb+wfnplot+ that allows the user to plot radial orbitals as functions of $\sqrt{r}$ or $r$. Assume that we have the $1s$, $2s$ and $2p$ orbitals
available in the file \verb+1Po.w+. To produce a Matlab/Octave file that plots the $2s$ and $2p$ orbitals follow the session below.

\begin{verbatim}
>>wfnplot


 ****************************************************
 Program wfnplot writes HF/MCHF radial wave functions
 to following output files:

 Matlab/GNU Octave file "octave_name.m"
 Xmgrace file "xmgrace_name.agr"

 Input file:  name.w

 To plot orbital: press enter
 To remove orbital: type "d" or "D" and press enter

                              Jorgen Ekman Jun 2015
 ****************************************************
 Name of state:
>>1Po

 To have r on x-axis: type "y" otherwise "n" for sqrt(r)
>>y
   1s = 
>>d	
   2s =
>>	
   2p =
>>	
\end{verbatim}

\noindent
In this case the output file is \verb+octave_1Po.m+. Starting Matlab or Octave and issuing the command
\verb+octave_1Po+ produces the plot below

\begin{center}
\begin{figure}[h!]
\includegraphics[clip=true,trim=30 170 50 180,angle=0,width=.85\textwidth]{orb.pdf}
  \caption{The $2s$ and $2p$ radial orbitals.}
\end{figure}
\end{center}
\end{document}


















*******************************************************************************
*         VIEW OUTPUT FILE 2SeCAS3.h                                          *
*         IN THIS CASE WE HAVE CONTRIBUTIONS ONLY FROM THE CONTACT TERM       *
*******************************************************************************

                     Hyperfine structure calculation

 2SeCAS3.l
 Nuclear dipole moment        1.00000000000 n.m.
 Nuclear quadrupole moment    1.00000000000 barns
 Nuclear spin                 1.00000000000

                              **

                    A factors in MHz

  J     J'        Orbital        Spin-dipole       Contact         Total

 1/2   1/2        0.0000000        0.0000000      180.0527840      180.0527840

                    B factors in MHz

  J     J'        Quadrupole

 1/2   1/2        0.0000000

                              **

                    Hyperfine parameters in a.u.

                    al             ad             ac             bq

                 0.0000000        0.0000000        2.8274233        0.0000000

                Electron density at the nucleus

 1/2   1/2      13.83982751
                              **
                              
                              
*******************************************************************************
*         VIEW MIXING COEFFICIENT FILE 2SeCAS3BREIT.j                         *
*         MIXING COEFFICENTS, ENERGIES, LANED gJ FACTORS ARE GIVEN            *
*******************************************************************************

  Li      Z =   3.0  NEL =   3   NCFG =     79


  2*J =    1  NUMBER =   1
   Ssms=   0.0000000000    g_J=   2.0023193044  g_JLS=   2.0023193044
     1    -7.473805156  1s(2).2s_2S
 0.97106712-0.21221165 0.05313322 0.04812769 0.06276643-0.00000010 0.00073827
-0.00000174-0.00000277 0.00271965 0.00000070-0.00115441 0.00000000-0.00000079
-0.00000001-0.00656366-0.00000019-0.00100410 0.00000031-0.00000074 0.00049817
 0.00000029 0.00000071 0.01265387 0.03890294-0.00000612 0.00000902-0.00591871
-0.00000397 0.00998694-0.00000226 0.00000653 0.00000000-0.02841430 0.00000001
 0.00965500-0.00000349 0.00000498-0.00835264 0.00000219-0.00000319-0.00834913
-0.00000145-0.00000174 0.00005105-0.00000010-0.00000003-0.00000006 0.00000001
 0.00116837 0.00000090 0.00226341-0.00000054 0.00000135 0.00000000 0.00000002
-0.00003397-0.00000004-0.00000001 0.00000000-0.00000002 0.00000000-0.00011586
 0.00000003-0.00000004 0.00000000-0.00202957 0.00000076-0.00000105 0.00178887
-0.00000053 0.00000064-0.00001235 0.00000000 0.00000000 0.00000001 0.00000000
 0.00000000 0.00000000
 **