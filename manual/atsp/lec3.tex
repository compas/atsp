\documentclass{amsart}
\usepackage{epsfig}

%sample command

%\begin{figure}
%\centerline{\epsfig{file=fig1.eps,height=3cm}}
%\end{figure}

\newcommand{\coh}{cohomology}
\newcommand{\Hom}{\operatorname{Hom}}
\newcommand{\id}{\operatorname{id}}
%\newcommand{\Ker}{\operatorname{Ker}}
%\newcommand{\Mc}{\overline{\mathcal{M}}}
%\newcommand{\gtmb}{\gtm^\bullet}
%\newcommand{\mgnb}[1]{\overline{\mathcal{M}}_{#1,n}}
%\newcommand{\mgn}[1]{\mathcal{M}_{#1,n}}
%\newcommand{\qis}{quasi-isomorphic}
%\newcommand{\rank}{\operatorname{rank}}
\newcommand{\tr}{\operatorname{tr}}
\newcommand{\Vect}{\operatorname{Vect}}

\newcommand{\nc}{\mathbb{C}}
\newcommand{\gtg}{\mathfrak{g}}
%\newcommand{\nq}{\mathbb{Q}}
\newcommand{\nr}{\mathbb{R}}
%\newcommand{\nz}{\mathbb{Z}}
\newcommand{\OO}{\mathcal{O}}

\newtheorem{thm}{Theorem}[section]
\newtheorem{lm}[thm]{Lemma}
\newtheorem{prop}[thm]{Proposition}
%\newtheorem*{cor}{Corollary}

\theoremstyle{definition}
\newtheorem{df}[thm]{Definition}
\newtheorem{ex}[thm]{Example}
\newtheorem{xca}{Exercise}

\theoremstyle{remark}
\newtheorem{rem}[thm]{Remark}
%\newtheorem*{ack}{Acknowledgment}

\numberwithin{equation}{section}

%    Absolute value notation
\newcommand{\abs}[1]{\lvert#1\rvert}

\begin{document}

\title[Lecture 3]{Lecture 3: Topological Quantum Field Theory and Frobenius Algebras}
\author{Alexander A. Voronov}
%\address{Department of Mathematics\\ M.I.T.,
%2-246\\ 77 Massachusetts Ave.\\ Cambridge, MA 02139-4307}
%\curraddr{}
%\email{voronov@math.mit.edu}
%\urladdr{http://www-math.mit.edu/~voronov/}
%\thanks{Research supported in part by an AMS Centennial Fellowship.}

%\subjclass{Primary 14H10; Secondary 32G15, 55P62}
\date{September 22, 1997}

\maketitle

\section{Topological quantum field theory and Frobenius algebras}

We saw in Lecture 2 that given a modular functor, the vector space
$V_\Sigma$ associated to a Riemann surface depends, up to a scalar,
only on the diffeomorphism class of the Riemann surface $\Sigma$. This
defines what is called a topological quantum field theory. We will be
mostly interested in the situation when the central charge $c = 0$ and
there is a specific operator $|\Sigma\rangle$ (as opposed to a vector
space of such operators) corresponding at least continuously to a
surface $\Sigma$. This all translates as follows into our language.

\begin{df}
A \emph{Topological Quantum Field Theory $($TQFT$)$} is a
correspondence of the same type as a Conformal Field Theory, see
Lecture 1, except that the conformal invariance axiom is replaced with
diffeomorphism invariance: the operator $|\Sigma\rangle$ must be
invariant under diffeomorphisms of the surface $\Sigma$ taking holes
to the corresponding holes and preserving the holomorphic coordinates
there. The normalization axiom is convenient to change so as the
operator corresponding to a cylinder (a sphere with two holes) is
equal to identity.
\end{df}

\begin{thm}[Folklore]
The structure of a TQFT based on vector space $V$ is equivalent to the
structure of a finite-dimensional Frobenius algebra on it. A
\emph{Frobenius algebra} structure on a vector space $V$ means the
structure of a commutative associative algebra with a unit and a
nondegenerate symmetric bilinear form $\langle,\rangle: V \otimes V
\to \nc$ which is invariant with respect to the multiplication: $\langle
ab, c\rangle = \langle a, bc \rangle$.
\end{thm}

\begin{proof}
Let $V$ be the state space of a TQFT. We would like to construct the
structure of a Frobenius algebra on $V$. Consider the following
Riemann surfaces and the corresponding operators, which we denote
$ab$, $\langle a ,b \rangle$, \emph{etc}.
\[
\begin{array}{ccll}
\parbox{2in}{\centerline{\epsfig{width=1.5in,height=.7in,file=pants.eps}}} &
 \quad \longmapsto \quad & V \otimes V \to V, & a \otimes b \mapsto ab
\smallskip\\
\parbox{2in}{\centerline{\epsfig{width=.9in,height=.7in,file=bublik.eps}}} &
 \quad \longmapsto \quad & V \otimes V \to \nc, & a \otimes b \mapsto
\langle a,b \rangle
\medskip\\
\parbox{2in}{\centerline{\epsfig{width=.9in,height=.7in,file=bublik.eps,angle=180}}} &
 \quad \longmapsto \quad & \psi: \nc \to V \otimes V
\medskip\\
\parbox{1.5in}{\centerline{\epsfig{width=1.2in,height=.3in,file=normal.eps}}} &
 \quad \longmapsto \quad & \id: V \to V
\medskip\\
\parbox{1.5in}{\centerline{\epsfig{width=.8in,height=.4in,file=cap1.eps}}} &
 \quad \longmapsto \quad & \tr: V \to \nc
\smallskip\\
\parbox{1.5in}{\centerline{\epsfig{width=.7in,height=.4in,file=cap2.eps}}} &
 \quad \longmapsto \quad & \nc \to V, & 1 \mapsto e
\end{array}
\]

We claim that these operators define the structure of a Frobenius
algebra on $V$. Indeed, the multiplication is commutative, because if
we interchange labels at the legs of a pair of pants we will get a
diffeomorphic Riemann surface. Therefore, the corresponding operator
$a \otimes b \mapsto ba$ will be equal to $ab$. Similarly, the
associativity $(ab)c = a(bc)$ of multiplication is based on the fact
that the following two surfaces are diffeomorphic:
\begin{equation}
\label{coherence}
\parbox{7cm}{\centerline{\parbox{3.6cm}{\epsfig{file=assoc1.eps,width=3.5cm}}
$\cong$
\ \ \
\parbox{3.5cm}{\epsfig{file=assoc2.eps,width=3.5cm}}}}
\end{equation}
The property $ae = ea = a$ of the unit element comes from the
diffeomorphism
\medskip

\centerline{\parbox{3cm}{\epsfig{file=unit.eps,width=2.5cm}} $\cong$
\ \ \ \ \ \ 
\parbox{3.5cm}{\epsfig{file=normal.eps,width=1.8cm}}}
\smallskip

Thus, we see that $V$ is a commutative associative unital
algebra. Now, the following diffeomorphism
\medskip

\centerline{\parbox{3cm}{\epsfig{file=bublik.eps,
width=1.5cm}} $\cong$
\ \ \ \ \ \ 
\parbox{3.5cm}{\epsfig{file=form.eps,width=1.8cm}}}
\medskip

\noindent
proves the identity $\langle a , b \rangle = \tr (ab)$, which, along
with the associativity, implies $\langle ab,c \rangle = \langle a,bc
\rangle$. The fact that the inner product $\langle \, , \, \rangle$ is
nondegenerate follows from the diffeomorphism
\bigskip

\centerline{\parbox{3cm}{\epsfig{file=nondeg.eps,
width=2cm}} $\cong$
\ \ \ \ \ \ 
\parbox{2.5cm}{\epsfig{file=normal.eps,width=1.8cm}},}
\medskip

\noindent
which implies that the composite mapping
\[
\begin{array}{ccccc}
V & \xrightarrow{\id\otimes \psi(1)} & V \otimes V \otimes V &
\xrightarrow{\langle,\rangle \otimes \id} &V,\\
v & \longmapsto & \sum_{i=1}^n v \otimes u_i \otimes v_i & \longmapsto &
\sum_{i=1}^n \langle v, u_i
\rangle v_i
\end{array}
\]
is equal to $\id : V \to V$. This also implies that $\dim V < \infty$,
because the $v_i$'s, $i = 1, \dots, n$, must span $V$. This completes
the construction of the structure of a Frobenius algebra on the state
space $V$ of a TQFT.

Conversely, if we have a finite-dimensional Frobenius algebra $V$, we
can define the structure of a TQFT on the vector space $V$ by (1)
cutting a Riemann surface down into pairs of pants, cylinders, and
caps; (2) defining the operators corresponding to those basic objects
using the multiplication (or its linear dual), the identity map, and
the unit element $e \in V$ (or the dual of the map $\nc \to V $, $ 1
\mapsto e$, as the trace functional), respectively; and (3) using the
sewing axiom. The fact that the composite operator is independent of
the way we cut down the surface follows from \eqref{coherence} and the
associativity of multiplication.
\end{proof}

\begin{xca}
Find a more elegant way (not using any elements of $V$) to show that
$\dim V < \infty$.
\end{xca}
%\bibliographystyle{/usr/local/lib/texmf/tex/latex/packages/amslatex/classes/amsplain}

%\bibliographystyle{amsalpha}

%\bibliography{operads}

\end{document}
